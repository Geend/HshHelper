% arara: pdflatex: { shell: true, draft: true }
% arara: makeglossaries
% arara: biber
% arara: pdflatex: { shell: true, synctex: true }
% arara: pdflatex: { shell: true, synctex: true }

\documentclass[12pt,DIV14,BCOR10mm,a4paper,parskip=half-,headsepline,headinclude,english,ngerman,bibliography=totocnumbered]{scrreprt}

\usepackage{hshhelper_base}

%%%%%%%%%%%%%%%%%%%%%%%%%%%%%%%%%%%%%%%%%%%%%%%%%%%%%%%%%%%%%%%%%%%%%%%%%%
\begin{document}    % hier gehts los
  \thispagestyle{empty} % Titelseite
\includegraphics[width=0.2\textwidth]{Wortmarke_WI_schwarz}

   {  ~ \sffamily
  \vfill
  {\Huge\bfseries Bewertung des Sicherheitsuntersuchungsberichts der Gruppe B}
  \bigskip

  {\Large
  Dennis Grabowski, Julius Zint, Philip Matesanz, Torben Voltmer \\[2ex]
  Masterprojekt \enquote{Entwicklung und Analyse einer sicheren \\Web-Anwendung} \\
  Wintersemester 18/19
 \\[5ex]
   \today }
}
 \vfill

  ~ \hfill
  \includegraphics[height=0.3\paperheight]{H_WI_Pantone1665}

\vspace*{-3cm}

\tableofcontents  % Inhaltsverzeichnis

\chapter{Stellungnahme zum Untersuchungsbericht von Gruppe B}

\section{HTML \texttt{blank}}

Wir denken, das Risiko, dass von dieser Lücke ausgeht, ist geringer als man einschätzt.
Entgegen der Darstellung von Gruppe B (\enquote{Dies gibt der aufrufenden, externen Webseite Zugriff auf die eigene Seite.}) kann die Zielseite keine völlige Kontrolle über die Ursprungsseite erlangen.
Sie kann lediglich umleiten, jedoch nicht auf die Inhalte der Ursprungsseite zugreifen.
Dazu kommt, dass man dafür JavaScript auf der jeweiligen Zielseite injizieren müsste.
Die Zielseiten \enquote{Bibliothek} und \enquote{ICMS} werden von uns allerdings als hinreichend sicher eingestuft.

Nichts desto trotz haben wir uns dazu entschieden, die Sicherheitlücke zu beheben.
Anstatt die Zielseite unmittelbar in einem neuen Tab zu öffnen, steuern wir auf eine von uns kontrollierte Zwischenseite an, diese kappt die Verbindung zur initialen HsH-Helfer-Seite, in dem das \texttt{opener}-Objekt auf \texttt{null} gesetzt wird.
Somit ist es der Zielseite nicht mehr möglich, den ursprünglichen Tab umzuleiten.

\section{Quotaänderungen werden nicht mehr geloggt}

Auch wenn dies, wie von Herrn Prof. Dr. Peine bestätigt wurde, keine Anforderung war, haben wir dies dennoch behoben.
Der Einwand der anderen Gruppe ist nämlich berechtigt: Ein halbes Logging ist kein gutes Logging.

\section{Einsicht der Dateimetadaten durch Administrator}

Praktisch betrachtet ist es einem Administrator jederzeit möglich, sich einer Gruppe hinzuzufügen, um somit auf die Dateien Zugriff zu erhalten.
Dennoch sehen wir das auch als Fehler der Anwendung, da ein Administrator keinen Zugriff auf diese Dateien hat, obwohl er die Metainformationen einsehen kann.
Daher haben wir auch dies behoben.

\section{Herausfinden des Quotalimits eines Nutzers durch Hochladen einer zu großen Datei}

Da würden wir genau so argumentieren wie Herr Prof. Dr. Peine in seiner E-Mail vom 22.01.2019 und sehen daher keinen Bedarf, hier Änderungen vorzunehmen.

Relevanter Auszug aus der E-Mail:

\blockquote{
Von: Peine, Holger \\
Gesendet: Dienstag, 22. Januar 2019 17:47 \\
An: Voltmer, Torben; Zint, Julius; Matesanz, Philip; Grabowski, Dennis; Gauggel, Alexander; Tietze, Torben; Bertram, Marvin; Volkert, Fabian  \\
Betreff: Meine Kommentare zum Untersuchungsbericht von Gruppe B \\

...

Ihr Befund, dass ein schreibberechtigter Benutzer herausfinden kann, wie viel
unbenutztes Speicherplatzlimit der Dateibesitzer noch hat, entspricht der
Anforderungsbeschreibung und stellt keine Sicherheitslücke dar: Anders ließe
sich das Schreiben durch andere Benutzer gar nicht realisieren.
}

\section{Google reCAPTCHA}

Die Aussage bezüglich des Captchas ist inhaltlich falsch, beim Login findet lediglich dann ein Zugriff auf die Google-Server statt, wenn ein Benutzerkonto und/oder eine Benutzer-IP \enquote{gesperrt} wurde.
Wir haben die Annahme getroffen, dass wir die Server von Google als vertrauenswürdig betrachten.
Diese Annahme sollte eventuell hinsichtlich ihrer Verfügbarkeit konkretisiert werden.

\section{Hochladen einer großen Datei zum DOS eines Nutzers}

Auch hier stimmen wir Herrn Prof. Dr. Peine's Argumentation (E-Mail vom 22.01.2019) bezüglich dieses Sachverhalts zu und sehen ebenfalls keinen Handlungsbedarf.

Relevanter Auszug aus der E-Mail:

\blockquote{
Von: Peine, Holger \\
Gesendet: Dienstag, 22. Januar 2019 17:47 \\
An: Voltmer, Torben; Zint, Julius; Matesanz, Philip; Grabowski, Dennis; Gauggel, Alexander; Tietze, Torben; Bertram, Marvin; Volkert, Fabian  \\
Betreff: Meine Kommentare zum Untersuchungsbericht von Gruppe B \\

...

Dieses Verhalten entspricht der Anforderungsbeschreibung und stellt
keine Sicherheitslücke dar: Anders ließe sich das Schreiben durch andere
Benutzer gar nicht realisieren.
}

\chapter{Stellungnahme zu den Kommentaren von Herrn Prof. Dr. Peine}

\section{Fehler 3.3.16}

Hierbei meinen wir ausschließlich den Umstand, dass eine leere Seite angezeigt wird und keinerlei Fehlermeldung an den Nutzer weitergegeben wird.
Das Verhalten der Applikation ist hier dennoch vollkommen korrekt.

\section{V1: Sperre eines Benutzerkontos}

Für uns stellt die Implementierung der Gegenseite eine eindeutige Sicherheitlücke dar: Nur weil ein Benutzer temporär gesperrt wird, hindert dies einen Angreifer nicht daran, dauerhafte, fehlerhafte Requests abzuschicken, um somit den Benutzer permanent auszusperren.

Dem stimmen Sie in dem von Ihnen formulierten \enquote{Projektablauf}-Dokument sogar zu, wodurch wir uns in unserer Behauptung bestätigt fühlen.

Auszug aus Kapitel 1 \enquote{Projektinhalt}, Seite 2 des \enquote{Projektablauf}-Dokumentes:

\blockquote{
Es gibt keine Vorgaben für die Performance der Anwendung. DoS-Angriffe auf Netzwerk- oder HTTP-Ebene (Requests in schneller Folge) müssen nicht abgewehrt werden, wohl aber DoS-Angriffe auf anwendungsinterne Ressourcen (z.B. \textbf{einen Benutzeraccount unbenutzbar machen durch wiederholte absichtlich falsche Passworteingabe, falls die Anwendung eine solche Reaktion implementiert).}
}

\section{V3: Cookie Reusing}

Es ist korrekt, dass ein Timeout alleine nicht ausreicht, dieses Problem zu beheben.
Daher haben wir in unserem Untersuchungsbericht daraufhin gewiesen, dass bei dem Sessionkonzept der gegnerischen Gruppe E-Mail-Adressen ausdrücklich nur einmal verwendet werden dürften.

\section{V10: Frühzeitiger Zugriff auf Account}

Wir sind davon ausgegangen, dass ein Benutzer sich ausschließlich mit seinem initialen Passwort erstmalig anmelden kann.
Da hier offensichtlich ein Missverständnis vorliegt, akzeptieren wir ihre Anmerkung.

\section{V13: Erzwungener Logout}

\section{5.9: \texttt{admin@admin.com} versus \texttt{admin@example.com}}

Tätsachlich ist die Domain \texttt{example.com} auch registriert, allerdings befindet sich diese im Besitz der IANA und ist explizit für Testzwecke vorgesehen.
Siehe RFC 2606 und 6761.

Wir argumentieren, dass die IANA vertrauenswürdiger ist als der momentane unbekannte Besitzer der Domain \texttt{admin.com}.
TODO snarky response.

\printbibliography

% Can be used to add a list of acronyms with their description
%\glsaddall
%\deftranslation{to=German}{Acronyms}{Abkürzungsverzeichnis}
%\deftranslation{to=German}{Glossary}{Glossar}
\printacronyms[title=Abkürzungsverzeichnis,toctitle=Abkürzungsverzeichnis]
\printglossary[title=Glossar,toctitle=Glossar,type=main]

%\addcontentsline{toc}{chapter}{\listfigurename}
% Insert list of figures, if a figure has been added to document
\iftotalfigures
  \listoffigures
\fi

%s\addcontentsline{toc}{chapter}{\listtablename}
% \listoftables       % Tabellenverzeichnis

\end{document}
