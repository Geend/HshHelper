% arara: pdflatex: { shell: true, draft: true }
% arara: makeglossaries
% arara: biber
% arara: pdflatex: { shell: true, synctex: true }
% arara: pdflatex: { shell: true, synctex: true }

\documentclass[12pt,DIV14,BCOR10mm,a4paper,parskip=half-,headsepline,headinclude,english,ngerman,bibliography=totocnumbered]{scrreprt}

\usepackage{hshhelper_base}

%%%%%%%%%%%%%%%%%%%%%%%%%%%%%%%%%%%%%%%%%%%%%%%%%%%%%%%%%%%%%%%%%%%%%%%%%%
\begin{document}    % hier gehts los
  \thispagestyle{empty} % Titelseite
\includegraphics[width=0.2\textwidth]{Wortmarke_WI_schwarz}

   {  ~ \sffamily
  \vfill
  {\Huge\bfseries Bewertung des Sicherheitsuntersuchungsberichts der Gruppe B}
  \bigskip

  {\Large
  Dennis Grabowski, Julius Zint, Philip Matesanz, Torben Voltmer \\[2ex]
  Masterprojekt \enquote{Entwicklung und Analyse einer sicheren \\Web-Anwendung} \\
  Wintersemester 18/19
 \\[5ex]
   \today }
}
 \vfill

  ~ \hfill
  \includegraphics[height=0.3\paperheight]{H_WI_Pantone1665}

\vspace*{-3cm}

\tableofcontents  % Inhaltsverzeichnis

\chapter{Stellungnahme zum Untersuchungsbericht von Gruppe B}

\section{HTML \texttt{blank}}

Wir denken, das Risiko, dass von dieser Luecke ausgeht, ist geringer als man einschaetzt.
Entgegen der Darstellung von Gruppe B (\enquote{Dies gibt der aufrufenden, externen Webseite Zugriff auf die eigene Seite.}) kann die Zielseite keine voellige Kontrolle ueber die Ursprungsseite erlangen.
Sie kann lediglich umleiten, jedoch nicht auf die Inhalte der Ursprungsseite zugreifen.
Dazu kommt, dass man dafuer JavaScript auf der jeweiligen Zielseite injizieren muesste.
Die Zielseiten \enquote{Bibliothek} und \enquote{ICMS} werden von uns allerdings als hinreichend sicher eingestuft.

Nichts desto trotz haben wir uns dazu entschieden, die Sicherheitluecke zu beheben.
Anstatt die Zielseite unmittelbar in einem neuen Tab zu oeffnen, steuern wir auf eine von uns kontrollierte Zwischenseite an, diese kappt die Verbindung zur initialen HsH-Helfer-Seite, in dem das \texttt{opener}-Objekt auf \texttt{null} gesetzt wurde.
Somit ist es der Zielseite nicht mehr moeglich, den urspruenglichen Tab umzuleiten.

\section{Quotaaenderungen werden nicht mehr geloggt}

Auch wenn dies, wie von Herrn Prof. Dr. Peine bestaetigt wurde, keine Anforderung war, haben wir dies dennoch behoben.
Der Einwand der anderen Gruppe ist dennoch berechtigt: Ein halbes Logging ist kein gutes Logging.

\section{Einsicht der Dateimetadaten durch Administrator}

Praktisch betrachtet ist es einem Administrator jederzeit moeglich, sich einer Gruppe hinzuzufuegen, um somit auf die Dateien Zugriff zu erhalten.
Dennoch sehen wir das auch als Fehler der Anwendung, da ein Administrator keinen Zugriff auf diese Dateien hat, obwohl er die Metainformationen einsehen kann.
Daher haben wir auch dies behoben.

\section{Herausfinden des Quotalimits eines Nutzers durch Hochladen einer zu grossen Datei}

Da wuerden wir genau so argumentieren wie Herr Prof. Dr. Peine in seiner E-Mail vom 22.01.2019 und sehen daher keinen Bedarf, hier Aenderungen vorzunehmen.

\section{Google reCAPTCHA}

Die Aussage bezueglich des Captchas ist inhaltlich falsch, beim Login findet lediglich dann ein Zugriff auf die Google-Server statt, wenn ein Benutzerkonto und/oder eine Benutzer-IP \enquote{gesperrt} wurde.
Wir haben die Annahme getroffen, dass wir die Server von Google als vertrauenswuerdig betrachten.
Diese Annahme sollte eventuell hinsichtlich ihrer Verfuegbarkeit konkretisiert werden.

\section{Hochladen einer grossen Datei zum DOS eines Nutzers}

Hier stimmen wir dem Inhalt der E-Mail vom 22.01.2019 geschrieben von Herrn Prof. Dr. Peine zu und sehen ebenfalls keinen Handlungsbedarf.

\chapter{Stellungnahme zu den Kommentaren von Herrn Prof. Dr. Peine}

\section{Fehler 3.3.16}

Hierbei meinen wir ausschliesslich den Umstand, dass eine leere Seite angezeigt wird und keinerlei Fehlermeldung an den Nutzer weitergegeben wird.
Das Verhalten der Applikation ist hier dennoch vollkommen korrekt.

\section{V1: Sperre eines Benutzerkontos}

Fuer uns stellt die Implementierung der Gegenseite eine eindeutige Sicherheitluecke dar: Nur weil ein Benutzer temporaer gesperrt wird, hindert dies einen Angreifer nicht daran, dauerhafte, fehlerhafte Requests abzuschicken, um somit den Benutzer permanent auszusperren.

TODO: Referenz auf Projektablauf Seite 2 - (Wir fuehlen uns bestaetigt durch ihre Aussage).

\section{V3: Cookie Reusing}

Es ist korrekt, dass ein Timeout alleine nicht ausreicht, dieses Problem zu beheben.
Daher haben wir in unserem Untersuchungsbericht ausgedruecklich daraufhin gewiesen, dass E-Mail-Adressen ausdruecklich nur einmal verwendet werden duerften.

\section{V10: Fruehzeitiger Zugriff auf Account}

Wir sind davon ausgegangen, dass ein Benutzer sich ausschliesslich mit seinem initialen Passwort erstmalig anmelden kann.
Da hier offensichtlich ein Missverstaendnis vorliegt, akzeptieren wir ihre Anmerkung.

\section{V13: Erzwungener Logout}

\section{5.9: \texttt{admin@admin.com} vs \texttt{admin@example.com}}

Taetsachlich ist die Domain \texttt{example.com} auch registriert, allerdings befindet sich diese im Besitz der IANA und ist explizit fuer Testzwecke vorgesehen.
Siehe RFC 2606 und 6761.

Wir argumentieren, dass die IANA vertrauenswuerdiger ist als der momentane unbekannte Besitzer der Domain \texttt{admin.com}.
TODO snarky response.

\printbibliography

% Can be used to add a list of acronyms with their description
%\glsaddall
%\deftranslation{to=German}{Acronyms}{Abkürzungsverzeichnis}
%\deftranslation{to=German}{Glossary}{Glossar}
\printacronyms[title=Abkürzungsverzeichnis,toctitle=Abkürzungsverzeichnis]
\printglossary[title=Glossar,toctitle=Glossar,type=main]

%\addcontentsline{toc}{chapter}{\listfigurename}
% Insert list of figures, if a figure has been added to document
\iftotalfigures
  \listoffigures
\fi

%s\addcontentsline{toc}{chapter}{\listtablename}
% \listoftables       % Tabellenverzeichnis

\end{document}
