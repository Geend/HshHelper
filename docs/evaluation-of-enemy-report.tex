% arara: pdflatex: { shell: true, draft: true }
% arara: makeglossaries
% arara: biber
% arara: pdflatex: { shell: true, synctex: true }
% arara: pdflatex: { shell: true, synctex: true }

\documentclass[12pt,DIV14,BCOR10mm,a4paper,parskip=half-,headsepline,headinclude,english,ngerman,bibliography=totocnumbered]{scrreprt}

\usepackage{hshhelper_base}

%%%%%%%%%%%%%%%%%%%%%%%%%%%%%%%%%%%%%%%%%%%%%%%%%%%%%%%%%%%%%%%%%%%%%%%%%%
\begin{document}    % hier gehts los
  \thispagestyle{empty} % Titelseite
\includegraphics[width=0.2\textwidth]{Wortmarke_WI_schwarz}

   {  ~ \sffamily
  \vfill
  {\Huge\bfseries Bewertung des Sicherheitsuntersuchungsberichts der Gruppe B}
  \bigskip

  {\Large
  Dennis Grabowski, Julius Zint, Philip Matesanz, Torben Voltmer \\[2ex]
  Masterprojekt \enquote{Entwicklung und Analyse einer sicheren \\Web-Anwendung} \\
  Wintersemester 18/19
 \\[5ex]
   \today }
}
 \vfill

  ~ \hfill
  \includegraphics[height=0.3\paperheight]{H_WI_Pantone1665}

\vspace*{-3cm}

\tableofcontents  % Inhaltsverzeichnis

\chapter{Stellungnahme zum Untersuchungsbericht von Gruppe B}

\section{HTML \texttt{blank}}

Entgegen der Darstellung von Gruppe B (\enquote{Dies gibt der aufrufenden, externen Webseite Zugriff auf die eigene Seite.}) kann die Zielseite keine völlige Kontrolle über das ursprüngliche Tab erlangen.
Sie kann in diesem lediglich eine Umleitung initiieren, jedoch nicht auf die Inhalte der Ursprungsseite zugreifen. Es wäre für diesen Angriff grundsätzlich erforderlich, dass der Angreifer Zugriff auf den Server der Zielseite erlangt, um diesen mit Schadcode zu infizieren. Die Zielseiten werden ausschließlich vom Administrator bestimmt und umfassen zum jetzigen Zeitpunkt ausschließlich die von der Hochschule betreuten Dienste \enquote{Bibliothek} und \enquote{ICMS}. Auch wenn die Aufgabenstellung diese Dienste nicht ausdrücklich als vertrauenswürdig bezeichnet, ergibt sich aus der Anforderungsbeschreibung zumindest ein gewisses Vertrauensverhältnis: Die Anforderungen verlangen ausdrücklich, dass der Hsh-Helfer die Zugangsdaten der Benutzer im Klartext übertragen soll. Ausgehend von dieser Anforderung, werden die Zielseiten von uns als hinreichend vertrauenswürdig eingestuft.

Dennoch haben wir die Sicherheitslücke behoben: Anstatt die Zielseite unmittelbar in einem neuen Tab zu öffnen, steuern wir eine von uns kontrollierte Zwischenseite an. Diese kappt die Verbindung zur initialen HsH-Helfer-Seite, indem das \texttt{opener}-Objekt auf \texttt{null} gesetzt wird.
Somit ist es der Zielseite nicht mehr möglich, den ursprünglichen Tab umzuleiten.

\section{Quotaänderungen werden nicht mehr geloggt}

Wir teilen die Auffassung von Herrn Prof. Dr. Peine aus der E-Mail vom 22.01.2019, dass diese Funktionalität nicht Teil der Anforderungen ist.

Die Lücke wurde dennoch durch uns behoben. Der Einwand der anderen Gruppe ist berechtigt: Ein halbes Logging ist kein gutes Logging.

\section{Einsicht der Dateimetadaten durch Administrator}

Ausweislich der Anforderungen ist es einem Administrator jederzeit möglich, auf die mit einer Gruppe geteilten Dateien zuzugreifen - er kann sich schließlich selbst jeder Gruppe hinzufügen.

Dennoch sehen wir dies auch als Fehler unserer Anwendung: Der Administrator hat keinen unmittelbaren Zugriff auf diese Dateien, also sollte er auch nicht ihre Meta-Informationen einsehen können.

Dieser Fehler wurde von uns behoben. Wir haben ebenfalls Testfälle erstellt, die diesen Umstand abdecken und sicherstellen, dass er nicht erneut auftreten wird.

\section{Herausfinden des Quotalimits eines Nutzers durch Hochladen einer zu großen Datei}

Wir teilen die Auffassung von Herrn Prof. Dr. Peine aus seiner E-Mail vom 22.01.2019 und sehen daher keinen Bedarf, hier Änderungen vorzunehmen.

Relevanter Auszug aus der E-Mail:

\blockquote{
Von: Peine, Holger \\
Gesendet: Dienstag, 22. Januar 2019 17:47 \\
An: Voltmer, Torben; Zint, Julius; Matesanz, Philip; Grabowski, Dennis; Gauggel, Alexander; Tietze, Torben; Bertram, Marvin; Volkert, Fabian  \\
Betreff: Meine Kommentare zum Untersuchungsbericht von Gruppe B \\

...

Ihr Befund, dass ein schreibberechtigter Benutzer herausfinden kann, wie viel
unbenutztes Speicherplatzlimit der Dateibesitzer noch hat, entspricht der
Anforderungsbeschreibung und stellt keine Sicherheitslücke dar: Anders ließe
sich das Schreiben durch andere Benutzer gar nicht realisieren.
}

\section{Google reCAPTCHA}

Die Aussage bezüglich des Captchas ist inhaltlich falsch, beim Login findet lediglich dann ein Zugriff auf die Google-Server statt, wenn ein Benutzerkonto und/oder eine Benutzer-IP \enquote{gesperrt} wurde.
Wir haben die Annahme getroffen, dass wir die Server von Google als vertrauenswürdig betrachten.
Diese Annahme sollte eventuell hinsichtlich ihrer Verfügbarkeit konkretisiert werden.

\section{Hochladen einer großen Datei zum DOS eines Nutzers}

Auch hier stimmen wir Herrn Prof. Dr. Peine's Argumentation (E-Mail vom 22.01.2019) bezüglich dieses Sachverhalts zu und sehen ebenfalls keinen Handlungsbedarf.

Relevanter Auszug aus der E-Mail:

\blockquote{
Von: Peine, Holger \\
Gesendet: Dienstag, 22. Januar 2019 17:47 \\
An: Voltmer, Torben; Zint, Julius; Matesanz, Philip; Grabowski, Dennis; Gauggel, Alexander; Tietze, Torben; Bertram, Marvin; Volkert, Fabian  \\
Betreff: Meine Kommentare zum Untersuchungsbericht von Gruppe B \\

...

Dieses Verhalten entspricht der Anforderungsbeschreibung und stellt
keine Sicherheitslücke dar: Anders ließe sich das Schreiben durch andere
Benutzer gar nicht realisieren.
}

\chapter{Stellungnahme zu den Kommentaren von Herrn Prof. Dr. Peine}

\section{Fehler 3.3.16}

Hierbei meinen wir ausschließlich den Umstand, dass eine leere Seite angezeigt wird und keinerlei Fehlermeldung an den Nutzer weitergegeben wird.
Das Verhalten der Applikation von Gruppe B entspricht der Anforderungsbeschreibung.

\section{V1: Sperre eines Benutzerkontos}

Für uns stellt die Implementierung der Gegenseite eine eindeutige Sicherheitlücke dar: Nur weil ein Benutzer lediglich \enquote{temporär} gesperrt wird, hindert dies einen Angreifer nicht daran, fortlaufend fehlerhafte Requests abzuschicken. Er kann die temporäre Sperre so für die Dauer seines Angriffs aufrecht erhalten. Wir haben Ihre Formulierung im \enquote{Projektablauf}-Dokument auch in diesem Sinne verstanden:

Auszug aus Kapitel 1 \enquote{Projektinhalt}, Seite 2 des \enquote{Projektablauf}-Dokumentes:

\blockquote{
Es gibt keine Vorgaben für die Performance der Anwendung. DoS-Angriffe auf Netzwerk- oder HTTP-Ebene (Requests in schneller Folge) müssen nicht abgewehrt werden, wohl aber DoS-Angriffe auf anwendungsinterne Ressourcen (z.B. \textbf{einen Benutzeraccount unbenutzbar machen durch wiederholte absichtlich falsche Passworteingabe, falls die Anwendung eine solche Reaktion implementiert).}
}

\section{V3: Cookie Reusing}

Es ist korrekt, dass ein Timeout alleine nicht ausreicht, um dieses Problem zu beheben.
Aus diesem Grund haben wir in unserem Untersuchungsbericht erwähnt, dass bei dem Sessionkonzept der gegnerischen Gruppe E-Mail-Adressen nur einmal verwendet werden dürfen.

\section{V10: Frühzeitiger Zugriff auf Account}

Wir sind davon ausgegangen, dass ein Benutzer sich ausschließlich mit seinem initialen Passwort erstmalig anmelden kann.
Da hier offensichtlich ein Missverständnis vorliegt, akzeptieren wir ihre Anmerkung.

\section{V13: Erzwungener Logout}

\section{5.9: \texttt{admin@admin.com} versus \texttt{admin@example.com}}

Wir haben die Anforderungen in dem Sinne verstanden, dass Ihre Wahl der Domain \texttt{example.com} mit ihrer Besonderen Bedeutung nach RFC 2606 bzw. 6761 zusammenhängt: Sie ist ausdrücklick für Testzwecke eingerichtet und freigeben. Außerdem befindet sie sich im Besitz der IANA - eine Organisation der wir größeres Vertrauen entgegenbringen als dem unbekannten Besitzer der Domain \texttt{admin.com}.
Da hier offensichtlich ein Missverständnis vorliegt, akzeptieren wir selbstverständlich Ihre Anmerkung.



\printbibliography

% Can be used to add a list of acronyms with their description
%\glsaddall
%\deftranslation{to=German}{Acronyms}{Abkürzungsverzeichnis}
%\deftranslation{to=German}{Glossary}{Glossar}
\printacronyms[title=Abkürzungsverzeichnis,toctitle=Abkürzungsverzeichnis]
\printglossary[title=Glossar,toctitle=Glossar,type=main]

%\addcontentsline{toc}{chapter}{\listfigurename}
% Insert list of figures, if a figure has been added to document
\iftotalfigures
  \listoffigures
\fi

%s\addcontentsline{toc}{chapter}{\listtablename}
% \listoftables       % Tabellenverzeichnis

\end{document}
