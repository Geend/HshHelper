% arara: pdflatex: { shell: true, draft: true }
% arara: makeglossaries
% arara: biber
% arara: pdflatex: { shell: true, synctex: true }
% arara: pdflatex: { shell: true, synctex: true }

\documentclass[fontsize=12pt,DIV=14,BCOR=10mm,a4paper,parskip=half-,headsepline,headinclude,english,ngerman,bibliography=totocnumbered]{scrreprt}

\usepackage{hshhelper_base}

\pagestyle{plain}

\makeatletter
\patchcmd{\scr@startchapter}{\if@openright\cleardoublepage\else\clearpage\fi}{}{}{}
\makeatother

%%%%%%%%%%%%%%%%%%%%%%%%%%%%%%%%%%%%%%%%%%%%%%%%%%%%%%%%%%%%%%%%%%%%%%%%%%
\begin{document}    % hier gehts los
  \thispagestyle{empty} % Titelseite
\includegraphics[width=0.2\textwidth]{Wortmarke_WI_schwarz}

   {  ~ \sffamily
  \vfill
  {\Huge\bfseries Sicherheitsrichtlinien}
  \bigskip

  {\Large
  Dennis Grabowski, Julius Zint, Philip Matesanz, Torben Voltmer \\[2ex]
  Masterprojekt \enquote{Entwicklung und Analyse einer sicheren \\Web-Anwendung} \\
  Wintersemester 18/19
 \\[5ex]
   \today }
}
 \vfill

  ~ \hfill
  \includegraphics[height=0.3\paperheight]{H_WI_Pantone1665}

\vspace*{-3cm}

\addto{\captionsgerman}{\renewcommand{\refname}{Literaturverzeichnis}}

\clearpage

\tableofcontents  % Inhaltsverzeichnis

\clearpage

\chapter{Annahmen}

\begin{itemize}
  \item Netzwerkverbindungen sind abhörsicher und von außen nicht beeinflussbar.
  \item Betriebssystem, Hardware, Datenbank sowie verwendete Bibliotheken enthalten keine sicherheitsrelevanten Fehler.
  \item Hashing-Algorithmus \enquote{bcrypt} macht es einem Angreifer wirtschaftlich unmöglich, evtl. erbeutete Password-Hashes durch Brute-Force oder Rainbow Tables in Plaintext zu verwandeln.
  \item Lösen eines reCAPTCHA ist für einen Angreifer wirtschaftlich undurchführbar.
  \item Die von Google-Servern eingebundenen reCAPTCHA JavaScript-Dateien werden niemals zur Code-Injection verwendet / Google's reCAPTCHA Server sind grund\-sätzlich vertrauenswürdig.
  \item Bibliotheksfunktion \texttt{java.security.SecureRandom} \autocite{JavaDocs.SecureRandom} erstellt Zufallszahlen, die kryptografisch sicher sind.
  \item Einstellung \texttt{ALLOW\_LITERALS=NONE} der Datenbank \enquote{h2} verhindert SQL Injections.
  \item \enquote{\gls{jwt}}-Format ist kryptografisch sicher zur Speicherung von Session- sowie Cookie-Daten.
  \item Die von JWT verwendete Signatur (HMAC-SHA256) verhindert, dass eine Manipulation von JWT-Cookies erfolgen kann.
  \item Applikation kann nur über die definierten Eintrittspunkte verwendet werden.
  \item Benutzer des HsH-Helpers verraten nicht ihr Passwort an andere.
  \item Initialer Benutzer \enquote{admin} ist vertrauenswürdig.
  \item Autogenerierte IDs der Datenbank \enquote{h2} sind aufsteigend und positiv. Sie werden nicht wiederverwendet, sofern eine ID wieder frei wird.
  \item Angreifer verfügen nur über einen begrenzten Pool an IP-Adressen, die \enquote{Anschaffung} großer Mengen von IP-Adressen für einen Login-Brute-Force ist unwirtschaftlich.
  \item Zertifikate sind grundsätzlich vertrauenswürdig. Certificate Authorities stellen keine Zertifikate für missbräuchliche Zwecke aus.
  \item Die Quelle der kryptografisch-sicheren Zufallszahlen versiegt nicht (\texttt{/dev/random} (Linux), Cryptography API: Next Generation (Windows) \autocite{Windows.SecureRandom}).
  \item Der gewählte Datentyp \texttt{Long} der Identifikatoren in der \enquote{h2-Datenbank} ist für den Benutzungskontext des HsH-Helpers ausreichend lang.
  \item Nutzer des HsH-Helpers verwenden moderne, aktuelle Browser, die spezifische sicherheitsrelevante HTTP-Header unterstützen (Browser nach 2013 für CSP).
  \item Alle Browser respektieren den HTTP-Header \texttt{Content-Type: application/octet\\-stream}.
  \item Unserer In-Memory-Datenbank geht niemals der Speicher aus, zumindest in Rahmen dieses Projekts.
  \item Nutzer unserer Applikation greifen nur innerhalb einer Sandbox auf eine Datei zu.
\end{itemize}

\renewcommand*{\chapterheadstartvskip}{\vspace*{\topskip}}

\chapter{Akteure}

\begin{itemize}
  \item Administratoren (A)
  \item Gruppenbesitzer (GO)
  \item Gruppenmitglied (GM)
  \item Dateibesitzer (DO)
  \item Authentisierter Benutzer mit Schreibeberechtigung auf eine Datei (DW)
  \item Authentisierter Benutzer mit Leseberechtigung auf eine Datei (DR)
  \item Authentisierter Benutzer (U+)
  \item Unauthentisierter Benutzer (U-)
  \item E-Mail Server (EM)
  \item Google reCAPTCHA Server (GR)
  \item Externer Netzdienst (N)
  \item Zwei-Faktor-Authentisierungsserver (2FAS)
\end{itemize}

Sofern nicht explizit anders geschildert, haben alle Akteure über dem \enquote{Authentisierten Benutzer} die selben Basisrechte wie diese Privilegienstufe.

\chapter{Eintrittspunkte}

\begin{itemize}
  \item Netzwerkschnittstellen
  \begin{itemize}
    \item \textbf{EP1:} HTTP (Port 80)  [U-, U+]
    \item \textbf{EP2:} SMTP (Port 587) [U-]
  \end{itemize}
  \item \textbf{EP3:} Eingebettetes Google reCAPTCHA JavaScript [U-, GR]
\end{itemize}

Sofern nicht explizit anders geschildert, geschieht ein Zugriff auf ein Asset oder das Durchführen einer Aktion über EP1.

\chapter{Assets}

\begin{itemize}
  \item Benutzeranmeldeinformationen
  \begin{itemize}
    \item \textbf{AS1:} Passworthash [-]
    \item \textbf{AS2:} Session [U+, GM, GO, A]
  \end{itemize}
  \item \textbf{AS3:} Gruppen [U+, GM, GO, A]
  \item \textbf{AS4:} Nutzerkonto [A]
  \item Dateien
  \begin{itemize}
    \item \textbf{AS5:} Dateiname [DR, DW, DO]
    \item \textbf{AS6:} Dateiinhalt [DR, DW, DO]
    \item \textbf{AS7:} Kommentar [DR, DW, DO]
    \item \textbf{AS8:} Zugriffsberechtigungen [DO]
  \end{itemize}
  \item Netzdienste
  \begin{itemize}
    \item \textbf{AS9:} Netzdienstanmelde-URL [A]
    \item \textbf{AS10:} Zugangsdaten eines Nutzers für einen externen Netzdienst [-]
  \end{itemize}
\end{itemize}

\chapter{Aktionen}

\begin{itemize}
  \item \textbf{AK1:} Einloggen [U-, GR] (Zusätzlich EP3)
  \item \textbf{AK2:} Ausloggen [U+]

  \item \textbf{AK3:} Nutzerkonto erstellen [A]
  \item \textbf{AK4:} Nutzerkonto löschen [A]
  \item \textbf{AK5:} Passwort zurücksetzen lassen [U-, EM, GR] (Zusätzlich EP2 \& EP3)
  \item \textbf{AK6:} Passwort nach Zurücksetzung anpassen [U+]
  \item \textbf{AK7:} Aktive Sessions anzeigen lassen [U+]
  \item \textbf{AK8:} Aktive Sessions zerstören [U+]

  \item \textbf{AK9:} Gruppe erstellen [U+]
  \item \textbf{AK10:} Gruppe löschen [GO, A]
  \item \textbf{AK11:} Nutzer zu einer Gruppe hinzufügen [GO, A]
  \item \textbf{AK12:} Nutzer aus einer Gruppe entfernen [GO, A]
  \item \textbf{AK13:} Gruppen anzeigen lassen [GM, GO, A]
  \item \textbf{AK14:} Mitglieder einer Gruppe sehen [GM, GO, A]

  \item \textbf{AK15:} Eine Datei hochladen [U+]
  \item \textbf{AK16:} Eine Datei löschen [DO]
  \item \textbf{AK17:} Eine Datei runterladen [DR, DO]
  \item \textbf{AK18:} Dateiinhalt und/oder -kommentar verändern [DW, DO]
  \item \textbf{AK19:} Dateien anzeigen lassen [U+]
  \item \textbf{AK20:} Nutzerberechtigungen einer Datei verwalten [DO]
  \item \textbf{AK21:} Gruppenberechtigungen einer Datei verwalten [DO]
  \item \textbf{AK22:} Informationen zu dem Speicherplatzlimit eines Nutzers anzeigen lassen [U+]
  \item \textbf{AK23:} Speicherplatzlimit eines Nutzers anpassen [A]
  \item \textbf{AK24:} Nutzer kann Session-Timeout einstellen [U+]
  \item \textbf{AK25:} Nutzer kann Passwort ändern [U+]
  \item \textbf{AK26:} Nutzer kann nach Dateien suchen [DR, DW, DO]

  \item \textbf{AK27:} Nutzer kann neuen Netzdienst hinzufügen [A]
  \item \textbf{AK28:} Nutzer kann bestehenden Netzdienst löschen [A]
  \item \textbf{AK29:} Nutzer kann Zugangsdaten für einen Netzdienst hinzufügen [U+]
  \item \textbf{AK30:} Nutzer kann hinterlegte Zugangsdaten für einen Netzdienst löschen [U+]
  \item \textbf{AK31:} Nutzer kann sich bei einem Netzdienst mit hinterlegten Zugangsdaten anmelden (Single Sign-On) [U+]

  \item \textbf{AK32:} Nutzer kann Zwei-Faktor-Authentisierung für Nutzerkonto aktivieren [U+]
  \item \textbf{AK33:} Nutzer kann Zwei-Faktor-Authentisierung für Nutzerkonto deaktivieren [U+]
\end{itemize}

\chapter{Richtlinien}
\newcolumntype{Y}{>{\centering\arraybackslash}X}

% htbp = HereTopBottomPage (priority of where to place the table in case previous placement fails)
\begin{table}[ht]
  \begin{tabularx}{\textwidth}{| l | Y | Y | Y | Y | Y | Y | Y | Y | Y |}
      \hline
      Aktion     & A & GO & GM & DO & DW & DR & U+ & U-    \\ \hline
      AK1        & \redxmark & \redxmark & \redxmark & \redxmark & \redxmark & \redxmark & \redxmark & \greencheckmark\\ \hline
      AK2        & \greencheckmark & \greencheckmark  & \greencheckmark & \greencheckmark & \greencheckmark  & \greencheckmark & \greencheckmark & \redxmark \\ \hline
      AK3        & \greencheckmark & \redxmark & \redxmark & \redxmark & \redxmark & \redxmark & \redxmark & \redxmark \\ \hline
      AK4        & \greencheckmark & \redxmark & \redxmark & \redxmark  & \redxmark & \redxmark & \redxmark & \redxmark \\ \hline
      AK5        & \redxmark & \redxmark & \redxmark & \redxmark  & \redxmark & \redxmark & \redxmark & \greencheckmark \\ \hline
      AK6        & \greencheckmark & \greencheckmark  & \greencheckmark & \greencheckmark & \greencheckmark  & \greencheckmark & \greencheckmark & \redxmark \\ \hline
      AK7        & \greencheckmark & \greencheckmark  & \greencheckmark & \greencheckmark & \greencheckmark  & \greencheckmark & \greencheckmark & \redxmark \\ \hline
      AK8        & \greencheckmark & \greencheckmark  & \greencheckmark & \greencheckmark & \greencheckmark  & \greencheckmark & \greencheckmark & \redxmark \\ \hline
      AK9        & \greencheckmark & \greencheckmark  & \greencheckmark & \greencheckmark & \greencheckmark  & \greencheckmark & \greencheckmark & \redxmark \\ \hline
      AK10       & \greencheckmark & \greencheckmark  & \redxmark & \redxmark & \redxmark & \redxmark & \redxmark & \redxmark \\ \hline
      AK11       & \greencheckmark & \greencheckmark  & \redxmark & \redxmark & \redxmark & \redxmark & \redxmark & \redxmark \\ \hline
      AK12       & \greencheckmark & \greencheckmark  & \redxmark & \redxmark & \redxmark & \redxmark & \redxmark & \redxmark \\ \hline
      AK13       & \greencheckmark & \greencheckmark  & \greencheckmark & \redxmark & \redxmark & \redxmark & \redxmark & \redxmark \\ \hline
      AK14       & \greencheckmark & \greencheckmark  & \greencheckmark & \redxmark  & \redxmark & \redxmark & \redxmark & \redxmark \\ \hline
      AK15       & \greencheckmark & \greencheckmark  & \greencheckmark & \greencheckmark & \greencheckmark  & \greencheckmark & \greencheckmark & \redxmark \\ \hline
      AK16       & \redxmark & \redxmark & \redxmark & \greencheckmark & \redxmark & \redxmark & \redxmark & \redxmark \\ \hline
      AK17       & \redxmark & \redxmark & \redxmark & \greencheckmark & \redxmark & \greencheckmark & \redxmark & \redxmark \\ \hline
      AK18       & \redxmark & \redxmark & \redxmark & \greencheckmark & \greencheckmark & \redxmark & \redxmark & \redxmark \\ \hline
      AK19       & \greencheckmark & \greencheckmark  & \greencheckmark & \greencheckmark & \greencheckmark  & \greencheckmark & \greencheckmark & \redxmark \\ \hline
      AK20       & \redxmark & \redxmark & \redxmark & \greencheckmark & \redxmark & \redxmark & \redxmark & \redxmark \\ \hline
      AK21       & \redxmark & \redxmark & \redxmark & \greencheckmark & \redxmark & \redxmark & \redxmark & \redxmark \\ \hline
      AK22       & \greencheckmark & \greencheckmark  & \greencheckmark & \greencheckmark & \greencheckmark  & \greencheckmark & \greencheckmark & \redxmark \\ \hline
      AK23       & \greencheckmark & \redxmark & \redxmark & \redxmark & \redxmark & \redxmark & \redxmark & \redxmark \\ \hline
      AK24       & \greencheckmark & \greencheckmark  & \greencheckmark & \greencheckmark & \greencheckmark  & \greencheckmark & \greencheckmark & \redxmark \\ \hline
      AK25       & \greencheckmark & \greencheckmark  & \greencheckmark & \greencheckmark & \greencheckmark  & \greencheckmark & \greencheckmark & \redxmark \\ \hline
      AK26       & \redxmark & \redxmark & \redxmark & \greencheckmark & \greencheckmark & \greencheckmark & \redxmark & \redxmark \\ \hline
  \end{tabularx}
\end{table}

\label{security-policies:new_policies}
\begin{itemize}
  \item Nutzer [U+] können nur mit dem System interagieren, wenn sie authentisiert sind (AK2-4, sowie AK6-21).
  \item Nutzer [GM, GO, A] können nur Gruppen sehen, dessen Mitglied sie sind (AK14).
  \item Ausschließlich Administratoren [A] können alle Gruppen sehen (AK14).
  \item Nutzer [GO] dürfen nur Mitglieder zu einer Gruppe hinzufügen, wenn sie der Besitzer dieser Gruppe sind (AK11).
  \item Ausschließlich Administratoren [A] können Mitglieder zu allen Gruppen hinzufügen (AK11).
  \item Nutzer [GO] dürfen nur Mitglieder aus einer Gruppe entfernen, wenn sie der Besitzer dieser Gruppe sind (AK12).
  \item Ausschließlich Administratoren [A] können Mitglieder (aber nicht den Besitzer) aus allen Gruppen entfernen (AK12).
  \item Kein Nutzer [U+] ist in der Lage, eine Gruppe zu erstellen (AK9), wenn der gewünschte Gruppenname (AS3) bereits von einer anderen Gruppe verwendet wird. Gruppennamen sind somit eindeutig.
  \item Kein Nutzer, auch nicht Administrator, [-] kann die Sessions anderer Nutzer betrachten oder zerstören (AK7-8).
  \item Passwörter eines Nutzer können von keinem Nutzer [-] ausgelesen werden.
  \item Ein Administrator [A] hat nur schreibenden Zugriff auf ein Nutzerkonto durch das Löschen des Nutzerkontos (AK4) oder durch Anpassen des Speicherplatzlimits eines Nutzers (AK23). Ihm ist nicht möglich, andere Informationen aus dem Nutzerkonto zu lesen oder zu ändern.
  \item Der initiale Benutzer \enquote{admin} kann nicht gelöscht werden.
  \item Ein Nutzer [U+] darf nur dann gelöscht werden, wenn er weder Besitzer der Gruppe Alle oder Administratoren ist.
  \item Ein Nutzer [U-] muss nur ein reCAPTCHA lösen, wenn er sich mehrmals hintereinander fehlerhaft eingeloggt hat (AK1).
  \item Die E-Mail eines Nutzerkontos ist einzigartig, so dass die Erstellung zweier Nutzerkonten mit der selben E-Mail-Adresse nicht möglich ist (AK3).
  \item Ein Nutzer [U-] muss zusätzlich ein reCAPTCHA lösen, um sein Passwort zurücksetzen lassen zu können (AK5).
  \item Ein Nutzer [U+] darf nur eine Datei hochladen (AK15), solange dessen Dateiname (AS5), -inhalt (AS6) und der Kommentar (AS7) zusammen nicht sein Speicherplatzlimit überschreiten.
  \item Ein Nutzer [DO] darf nur eine Datei löschen (AK16), wenn er dessen Besitzer ist.
  \item Ein Nutzer [DR] darf beim lesenden Zugriff auf eine Datei (AK17) keine Informationen verändern.
  \item Ein Nutzer [DW, DO] darf beim Verändern des Dateiinhalts oder des Kommentars (AK18) niemals den Dateinamen anpassen.
  \item Ein Nutzer [U+] darf sich nur Dateien anzeigen lassen (AK19), zu welchen mindestens zum Lesen oder Schreiben berechtigt ist.
  \item Ein Nutzer [DO] darf die Nutzer- und Gruppenberechtigungen einer Datei jederzeit anpassen (AK20, 21); bedeutet löschen, hinzufügen, verändern.
  \item Ein Nutzer [U+] darf sich nur zu seinem eigenem Speicherplatzlimit informieren (AK22) und hat keinen Zugriff auf die Speicherplatzlimit anderer Nutzerkonten.
  \item Ein Nutzer [U+] darf seinen eigenen Session-Timeout (AK24) einstellen.
  \item Ein Nutzer [U+] darf sein eigenes Passwort (AK25) einstellen.
  \item Nur der Dateibesitzer [DO] darf schreibend auf den Dateinamen (AS5) zugreifen, alle anderen [DR, DW] nur lesend.
\end{itemize}

\printbibliography

% Can be used to add a list of acronyms with their description
% \glsaddall
%\deftranslation{to=German}{Acronyms}{Abkürzungsverzeichnis}
%\deftranslation{to=German}{Glossary}{Glossar}
\printacronyms[title=Abkürzungsverzeichnis,toctitle=Abkürzungsverzeichnis]
\printglossary[title=Glossar,toctitle=Glossar,type=main]

% Insert list of figures, if a figure has been added to document
\iftotalfigures
  \listoffigures
\fi
%s\addcontentsline{toc}{chapter}{\listtablename}
% \listoftables       % Tabellenverzeichnis

\end{document}
