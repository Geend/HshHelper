% arara: pdflatex: { shell: true, draft: true }
% arara: makeglossaries
% arara: biber
% arara: pdflatex: { shell: true, synctex: true }
% arara: pdflatex: { shell: true, synctex: true }

\documentclass[fontsize=12pt,DIV=14,BCOR=10mm,a4paper,parskip=half-,english,ngerman,bibliography=totocnumbered]{scrreprt}

\usepackage{hshhelper_base}

\makeatletter
\patchcmd{\scr@startchapter}{\if@openright\cleardoublepage\else\clearpage\fi}{}{}{}
\makeatother

\pagestyle{plain}

%%%%%%%%%%%%%%%%%%%%%%%%%%%%%%%%%%%%%%%%%%%%%%%%%%%%%%%%%%%%%%%%%%%%%%%%%%
\begin{document}    % hier gehts los
  \thispagestyle{empty} % Titelseite
\includegraphics[width=0.2\textwidth]{Wortmarke_WI_schwarz}

   {  ~ \sffamily
  \vfill
  {\Huge\bfseries Sicherheitsrichtlinien}
  \bigskip

  {\Large
  Dennis Grabowski, Julius Zint, Philip Matesanz, Torben Voltmer \\[2ex]
  Masterprojekt \enquote{Entwicklung und Analyse einer sicheren \\Web-Anwendung} \\
  Wintersemester 18/19\makeatletter
  \patchcmd{\scr@startchapter}{\if@openright\cleardoublepage\else\clearpage\fi}{}{}{}
  \makeatother
 \\[5ex]
   \today }
}
 \vfill

  ~ \hfill
  \includegraphics[height=0.3\paperheight]{H_WI_Pantone1665}

\vspace*{-3cm}

\addto{\captionsgerman}{\renewcommand{\refname}{Literaturverzeichnis}}

\clearpage

\tableofcontents  % Inhaltsverzeichnis

\clearpage

\chapter{Annahmen}

\begin{itemize}
  \item Netzwerkverbindungen sind abhörsicher und von außen nicht beeinflussbar.
  \item Betriebssystem, Hardware sowie verwendete Bibliotheken enthalten keine sicherheitsrelevanten Fehler.
  \item Hashing-Algorithmus \enquote{bcrypt} macht es einem Angreifer wirtschaftlich unmöglich, evtl. erbeutete Password-Hashes durch Brute-Force oder Rainbow Tables in Plaintext zu verwandeln.
  \item Lösen eines Recaptchas ist für einen Angreifer wirtschaftlich undurchführbar.
  \item Die von Google-Servern eingebundenen Recaptcha Java-Script Dateien werden niemals zur Code-Injection verwendet / Google's reCAPTCHA Server sind grund\-sätzlich vertrauenswürdig.
  \item Bibliotheksfunktion \texttt{java.security.SecureRandom} \autocite{JavaDocs.SecureRandom} erstellt Zufallszahlen, die kryptografisch sicher sind.
  \item Einstellung \texttt{ALLOW\_LITERALS=NONE} der Datenbank \enquote{h2} verhindert SQL Injections.
  \item \enquote{\gls{jwt}}-Format ist kryptografisch sicher zur Speicherung von Session- sowie Cookie-Daten.
  \item Die von JWT verwendete Signatur (HMAC-SHA256) verhindert, dass eine Manipulation von JWT-Cookies erfolgen kann.
  \item Applikation kann nur über die definierten Eintrittspunkte verwendet werden.
  \item Benutzer des HsH-Helfers verraten nicht ihr Passwort an andere.
  \item Initialer Benutzer \enquote{admin} ist vertrauenswürdig.
  \item Autogenerierte IDs der Datenbank \enquote{h2} sind aufsteigend und positiv. Sie werden nicht wiederverwendet, sofern eine ID wieder frei wird.
  \item Angreifer verfügen nur über einen begrenzten Pool an IP-Addressen, die \enquote{Anschaffung} großer Mengen von IP-Adressen für einen Login-Brute-Force ist unwirtschaftlich.
  \item Zertifikate sind grundsätzlich vertrauenswürdig. Certificate Authorities stellen keine Zertifikate für missbräuchliche Zwecke aus.
  \item Die Quelle der kryptografisch-sicheren Zufallszahlen versiegt nicht (\texttt{/dev/random} (Linux), Cryptography API: Next Generation (Windows) \autocite{Windows.SecureRandom}).
  \item Der gewählte Datentyp \texttt{Long} der Identifikatoren in der \enquote{h2-Datenbank} ist für den Benutzungskontext des HsH-Helfers ausreichend lang.
\end{itemize}

\chapter{Akteure}

\begin{itemize}
  \item Administratoren (A)
  \item Gruppenbesitzer (GO)
  \item Gruppenmitglied (GM)
  \item Authentisierter Benutzer (U+)
  \item Unauthentisierter Benutzer (U-)
  \item E-Mail Server (EM)
  \item Google reCAPTCHA Server (GR)
\end{itemize}

\chapter{Eintrittspunkte}

\begin{itemize}
  \item Netzwerkschnittstellen
  \begin{itemize}
    \item \textbf{EP1:} HTTP (Port 80)  [U-, U+, GM, GO, A]
    \item \textbf{EP2:} SMTP (Port 587) [U-]
  \end{itemize}
  \item \textbf{EP3:} Eingebettetes Google reCAPTCHA JavaScript [U-, GR]
\end{itemize}

Sofern nicht anders geschildert, geschieht ein Zugriff auf ein Asset oder das Durchführen einer Aktion über EP1.

\chapter{Assets}

\begin{itemize}
  \item Benutzeranmeldeinformationen
  \begin{itemize}
    \item \textbf{AS1:} Passworthash [-]
    \item \textbf{AS2:} Session [U+, GM, GO, A]
  \end{itemize}
  \item \textbf{AS3:} Gruppen [U+, GM, GO, A]
  \item \textbf{AS4:} Nutzerkonto [A]
\end{itemize}

\chapter{Aktionen}

\begin{itemize}
  \item \textbf{AK1:} Einloggen [U-, GR] (Zusätzlich EP3)
  \item \textbf{AK2:} Ausloggen [U+, GM, GO, A]
  \item \textbf{AK3:} Nutzerkonto erstellen [A]
  \item \textbf{AK4:} Nutzerkonto löschen [A]
  \item \textbf{AK5:} Passwort zurücksetzen lassen [U-, EM, GR] (Zusätzlich EP2 \& EP3)
  \item \textbf{AK6:} Passwort nach Zurücksetzung anpassen [U+, GM, GO, A]
  \item \textbf{AK7:} Aktive Sessions anzeigen lassen [U+, GM, GO, A]
  \item \textbf{AK8:} Aktive Sessions zerstören [U+, GM, GO, A]
  \item \textbf{AK9:} Gruppe erstellen [U+, GM, GO, A]
  \item \textbf{AK10:} Gruppe löschen [GO, A]
  \item \textbf{AK11:} Nutzer zu einer Gruppe hinzufügen [GO, A]
  \item \textbf{AK12:} Nutzer aus einer Gruppe entfernen [GO, A]
  \item \textbf{AK13:} Gruppen anzeigen lassen [U+, GM, GO, A]
  \item \textbf{AK14:} Mitglieder einer Gruppe sehen [GM, GO, A]
\end{itemize}

\chapter{Richtlinien}
\newcolumntype{Y}{>{\centering\arraybackslash}X}

% htbp = HereTopBottomPage (priority of where to place the table in case previous placement fails)
\begin{table}[ht]
  \begin{tabularx}{\textwidth}{| l | Y | Y | Y | Y |}
      \hline
      Aktion     & Administrator     & Gruppenbesitzer & Gruppenmitglied & Nutzer     \\ \hline
      AK1        & \redxmark & \redxmark & \redxmark & \redxmark \\ \hline
      AK2        & \greencheckmark & \greencheckmark  & \greencheckmark & \greencheckmark \\ \hline
      AK3        & \greencheckmark & \redxmark & \redxmark & \redxmark \\ \hline
      AK4        & \greencheckmark & \redxmark & \redxmark & \redxmark \\ \hline
      AK5        & \redxmark & \redxmark & \redxmark & \redxmark \\ \hline
      AK6        & \greencheckmark & \greencheckmark  & \greencheckmark & \greencheckmark \\ \hline
      AK7        & \greencheckmark & \greencheckmark  & \greencheckmark & \greencheckmark \\ \hline
      AK8        & \greencheckmark & \greencheckmark  & \greencheckmark & \greencheckmark \\ \hline
      AK9        & \greencheckmark & \greencheckmark  & \greencheckmark & \greencheckmark \\ \hline
      AK10        & \greencheckmark & \greencheckmark  & \redxmark & \redxmark \\ \hline
      AK11        & \greencheckmark & \greencheckmark  & \redxmark & \redxmark \\ \hline
      AK12        & \greencheckmark & \greencheckmark  & \redxmark & \redxmark \\ \hline
      AK13       & \greencheckmark & \greencheckmark  & \greencheckmark & \greencheckmark \\ \hline
      AK14       & \greencheckmark & \greencheckmark  & \greencheckmark & \redxmark \\ \hline
  \end{tabularx}
\end{table}

\label{security-policies:new_policies}
\begin{itemize}
  \item Nutzer [U+, GM, GO, A] können nur mit dem System interagieren, wenn sie authentisiert sind (AK2-4, sowie AK6-14).
  \item Nutzer [U+, GM, GO, A] können nur Gruppen sehen, dessen Mitglied sie sind (AK14).
  \item Ausschließlich Administratoren [A] können alle Gruppen sehen (AK14).
  \item Nutzer [GO] dürfen nur Mitglieder zu einer Gruppe hinzufügen, wenn sie der Besitzer dieser Gruppe sind (AK11).
  \item Ausschließlich Administratoren [A] können Mitglieder zu allen Gruppen hinzufügen (AK11).
  \item Nutzer [GO] dürfen nur Mitglieder aus einer Gruppe entfernen, wenn sie der Besitzer dieser Gruppe sind (AK12).
  \item Ausschließlich Administratoren [A] können Mitglieder (aber nicht den Besitzer) aus allen Gruppen entfernen (AK12).
  \item Kein Nutzer, auch nicht Administrator, [-] kann die Sessions anderer Nutzer betrachten oder zerstören (AK7-8).
  \item Passwörter eines Nutzer können von keinem Nutzer [-] ausgelesen werden.
  \item Ein Administrator [A] hat nur schreibenden Zugriff auf ein Nutzerkonto durch das Löschen (AK4). Ihm ist nicht möglich, andere Informationen aus dem Nutzerkonto zu lesen oder zu ändern.
  \item Ein Nutzer [-] muss nur ein reCAPTCHA lösen, wenn er sich mehrmals hintereinander fehlerhaft eingeloggt hat (AK1).
  \item Die E-Mail eines Nutzerkontos ist einzigartig, so dass die Erstellung zweier Nutzerkonten mit der selben E-Mail-Adresse nicht möglich ist (AK3).
  \item Ein Nutzer [U-] muss zusätzlich ein reCAPTCHA lösen, um sein Passwort zurücksetzen lassen zu können (AK5).
  \item Ein Nutzer [U+, GM, GO, A] darf nur dann gelöscht werden, wenn er weder Owner der Gruppe Alle oder Administratoren ist.
\end{itemize}

\printbibliography

% Can be used to add a list of acronyms with their description
% \glsaddall
%\deftranslation{to=German}{Acronyms}{Abkürzungsverzeichnis}
%\deftranslation{to=German}{Glossary}{Glossar}
\printacronyms[title=Abkürzungsverzeichnis,toctitle=Abkürzungsverzeichnis]
\printglossary[title=Glossar,toctitle=Glossar,type=main]

%s\addcontentsline{toc}{chapter}{\listtablename}
% \listoftables       % Tabellenverzeichnis

\end{document}
