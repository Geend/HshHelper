% arara: pdflatex: { shell: true, draft: true }
% arara: makeglossaries
% arara: biber
% arara: pdflatex: { shell: true, synctex: true }
% arara: pdflatex: { shell: true, synctex: true }

\documentclass[12pt,DIV14,BCOR10mm,a4paper,parskip=half-,headsepline,headinclude,english,ngerman,bibliography=totocnumbered]{scrreprt}

\usepackage{hshhelper_base}

\makeatletter
\patchcmd{\scr@startchapter}{\if@openright\cleardoublepage\else\clearpage\fi}{}{}{}
\makeatother

\pagestyle{plain}

%%%%%%%%%%%%%%%%%%%%%%%%%%%%%%%%%%%%%%%%%%%%%%%%%%%%%%%%%%%%%%%%%%%%%%%%%%
\begin{document}    % hier gehts los
  \thispagestyle{empty} % Titelseite
\includegraphics[width=0.2\textwidth]{Wortmarke_WI_schwarz}

   {  ~ \sffamily
  \vfill
  {\Huge\bfseries Anforderungsbeschreibung der von uns implementierten Zusatzfunktionalitäten}
  \bigskip

  {\Large
  Dennis Grabowski, Julius Zint, Philip Matesanz, Torben Voltmer \\[2ex]
  Masterprojekt \enquote{Entwicklung und Analyse einer sicheren \\Web-Anwendung} \\
  Wintersemester 18/19
 \\[5ex]
   \today }
}
 \vfill

  ~ \hfill
  \includegraphics[height=0.3\paperheight]{H_WI_Pantone1665}

\vspace*{-3cm}

\clearpage

\tableofcontents  % Inhaltsverzeichnis

\clearpage

\chapter{Benutzerverwaltung}

\renewcommand*{\chapterheadstartvskip}{\vspace*{22pt}}

\section{Zwei-Faktor Authentisierung}

\begin{enumerate}
    \item Ein angemeldeter Nutzer kann die Zwei-Faktor Authentisierung aktivieren.
    \item Ein angemeldeter Nutzer, der zuvor die Zwei-Faktor Authentisierung aktiviert hat, kann diese auch wieder deaktivieren.
    \item Hat ein angemeldeter Nutzer die Zwei-Faktor Authentisierung aktiviert, so muss er diese bei jedem Login-Versuch verwenden.
    \item Sollte ein Nutzer seinen zweiten Faktor verlieren, muss er einem Administrator Bescheid geben, so dass dieser die Zwei-Faktor Authentisierung deaktiviert, sofern dieser den Akteur für vertrauenswürdig hält.
    \item Die Zwei-Faktor Authentisierung ist so implementiert, dass es keinem statistisch möglich sein sollte, das Shared Secret zu erraten.
    \begin{itemize}
        \item Wir nutzen hierfür eine Implementation basierend auf einem Time-Based-One-Time-Password.
    \end{itemize}
    \item Administratoren haben eine neue Administratorfunktion, mit welcher sie die Zwei-Faktor-Authentisierung eines anderen Nutzerkontos deaktivieren können. Diese Funktion wurde hinzugefügt für den Fall, dass ein Nutzer seinen zweiten Faktor verloren hat, da er sonst nicht mehr einloggen kann.
\end{enumerate}

\section{Passwort ändern}

\begin{enumerate}
    \item Ein angemeldeter Nutzer soll in der Lage sein, sein Passwort ändern zu könnnen.
    \item Für eine Passwortänderung muss ein angemeldeter Nutzer folgende Daten angeben:
    \begin{enumerate}
        \item Aktuelles Passwort.
        \item Neues Passwort.
        \item Erneute Eingabe des neuen Passworts zur Bestätigung dieses Passworts.
    \end{enumerate}
\end{enumerate}

\chapter{Sessions verwalten}

\begin{enumerate}
    \item Ein angemeldeter Nutzer kann seine eigenen, aktiven Sessions betrachten.
    \item Ein angemeldeter Nutzer kann seine eigenen, aktiven Sessions invalidieren.
    \item Ein angemeldeter Nutzer kann zusätzlich alle bisherigen Logins betrachten.
    \item Ein angemeldeter Nutzer kann seinen Session-Timeout einstellen:
    \begin{enumerate}
        \item Der Session-Timeout soll in Minuten angegeben werden.
        \item Die absolute Mindestdauer einer Session beträgt 5 Minuten. Das ist auch der voreingestellte Wert.
        \item Die absolute Maximaldauer einer Session beträgt einen Tag (1440 Minuten).
    \end{enumerate}
\end{enumerate}

\chapter{Datei-Austausch}

\begin{enumerate}
    \item Ein angemeldeter Nutzer kann zusätzlich zu einer Datei, auf die er Zugriff hat, sehen, welcher Nutzer die Datei als letztes überschrieben hat.
    \item Dateien sollen visuell hervorgehoben werden, um die Vertrauenswürdigkeit besser darstellen zu können:
    \begin{enumerate}
        \item Eine Datei soll grün hinterlegt werden, wenn Diese Datei zuletzt durch den Nutzer beschrieben wurde.
        \item Eine Datei soll rot hinterlegt werden, wenn diese Datei zuletzt durch einen \textbf{anderen} Nutzer beschrieben wurde.
    \end{enumerate}
    \item Ein angemeldeter Nutzer kann nach Dateien suchen:
    \begin{enumerate}
        \item Diese Suchfunktion soll einem Nutzer in jeder Ansicht bereit stehen.
        \item Er kann nur nach Dateien suchen, auf die er durch eine Nutzer- oder Gruppenberechtigung Zugriff hat. Andere Dateien darf er nicht durch die Suche finden.
    \end{enumerate}
\end{enumerate}

\printbibliography

% Can be used to add a list of acronyms with their description
%\glsaddall
%\deftranslation{to=German}{Acronyms}{Abkürzungsverzeichnis}
%\deftranslation{to=German}{Glossary}{Glossar}
\printacronyms[title=Abkürzungsverzeichnis,toctitle=Abkürzungsverzeichnis]
\printglossary[title=Glossar,toctitle=Glossar,type=main]

%\addcontentsline{toc}{chapter}{\listfigurename}
% Insert list of figures, if a figure has been added to document
\iftotalfigures
  \listoffigures
\fi

%s\addcontentsline{toc}{chapter}{\listtablename}
% \listoftables       % Tabellenverzeichnis

\begin{appendices}

\end{appendices}

\end{document}
