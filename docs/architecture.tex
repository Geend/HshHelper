\documentclass[12pt,DIV14,BCOR10mm,a4paper,twoside,parskip=half-,headsepline,headinclude,english,ngerman,bibliography=totocnumbered]{scrreprt}
% Grundgröße 12pt, zweiseitig
\usepackage[headsepline,automark]{scrpage2}
\KOMAoptions{headinclude}
% Seitenköpfe automatisch
% Sprachpaket für Deutsch (Umlaute, Trennung,deutsche Überschriften)
\usepackage{babel}
\usepackage{blindtext}
%Graphikeinbindung, Hyperref (alles klickbar, Bookmarks),
%Math. Symbole aus AmsTeX
\usepackage{graphicx,hyperref,amssymb,url,atbegshi,etoolbox}
\usepackage[backend=biber,natbib=true, style=numeric, citestyle=numeric, maxnames=7, minnames=1, maxcitenames=2, url=true]{biblatex}
% Umlaute und über Tastatur einzugeben
\usepackage[utf8]{inputenc}
% add table of contents and other tables automatically
\usepackage[nottoc]{tocbibind}
%\usepackage[toc,page]{appendix}

\usepackage[autostyle=true,german=quotes]{csquotes}
%% json related packages
% dont use bera, can't get it at the work PC
%\usepackage{bera} % optional: just to have a nice mono-spaced font
\usepackage{listings}
\usepackage{xcolor}
\usepackage{textcomp}

%%% Change settings for epigraphs
%%% Change chapter format, if needed.
% customize chapter format:
% \KOMAoption{headings}{twolinechapter}
% \renewcommand*\chapterformat{\thechapter\autodot}

% customize dictum format:
\setkomafont{dictumtext}{\itshape\small}
\setkomafont{dictumauthor}{\normalfont}
\renewcommand*\dictumwidth{\linewidth}
\renewcommand*\dictumauthorformat[1]{--- #1}
\renewcommand*\dictumrule{}

\renewcommand{\lstlistingname}{Quellcode}

\makeatletter
\makeatother

% create glossaries
\usepackage[toc,section=chapter,nopostdot,acronym,style=list]{glossaries}
% style & idea copied from: https://tex.stackexchange.com/questions/8946/how-to-combine-acronym-and-glossary
\usepackage{xparse}
\DeclareDocumentCommand{\newdualentry}{ O{} O{} m m m m } {

  %%% The glossary entry the acronym links to
  \newglossaryentry{g-#3}{name={#5},
      description={#6},#1}

  %%% define the acronym and use the see= option
  \newglossaryentry{#3}{type=\acronymtype, name={#4}, description={#5}, first={#5 (#4)\glsadd{g-#3}}, see=[Glossareintrag:]{g-#3}}
}

\makeglossaries

%\loadglsentries[main]{acronyms.tex}

% Festlegung Kopf- und Fußzeile
\defpagestyle{meinstil}{%
{\headmark \hfill}
{\hfill \headmark}
{\hfill \headmark\hfill }
(\textwidth,.4pt)
}{%
(\textwidth,.4pt)
{\pagemark\hfill Dennis Grabowski, Julius Zint, Philip Matesanz, Torben Voltmer}
{Version 1.0 vom \today \hfill \pagemark}
{Version 1.0 vom \today\hfill\pagemark}
}
\pagestyle{meinstil}

\raggedbottom
\renewcommand{\topfraction}{1}
\renewcommand{\bottomfraction}{1}

\bibliography{literature.bib}
\graphicspath{{./images/}}

%%%%%%%%%%%%%%%%%%%%%%%%%%%%%%%%%%%%%%%%%%%%%%%%%%%%%%%%%%%%%%%%%%%%%%%%%%
\begin{document}    % hier gehts los
  \thispagestyle{empty} % Titelseite
\includegraphics[width=0.2\textwidth]{Wortmarke_WI_schwarz}

   {  ~ \sffamily
  \vfill
  {\Huge\bfseries Architekturbeschreibung}
  \bigskip

  {\Large
  Dennis Grabowski, Julius Zint, Philip Matesanz, Torben Voltmer \\[2ex]
  Masterprojekt \enquote{Entwicklung und Analyse einer sicheren Web-Anwendung} \\
  Wintersemester 18/19
 \\[5ex]
   \today }
}
 \vfill

  ~ \hfill
  \includegraphics[height=0.3\paperheight]{H_WI_Pantone1665}

\vspace*{-3cm}

\tableofcontents  % Inhaltsverzeichnis

\chapter{Architekturaufbau}

Klassendiagramm (High-Level-Ansicht)
Ggf. Sequenzdiagramme fuer wichtige Ablaeufe
Verantwortung der jeweiligen Klassen auflisten
Welche Klasse macht mit welcher Klasse was?

\chapter{Sicherheitsmassnahmen}

\section{Kryptografisch relevanten Informationen}
Welche Algorithmen verwenden wir?
Wie stellen diese sicher, dass Informationen usw nicht geklaut/geknackt/abgehoert/entwendet werden koennen?
Wie sieht es mit unserer Schluesselverwaltung aus?
Wie nutzen wir unsere Cookies? Was machen unsere Cookies genau?
Wann setzen wir sie?


\subsection{HshHelper Passwörter}
Die Passwörter, die Benutzer verwenden um sich bei HshHelper zu authentisieren werden grundsätzlich nur als Hash persistiert. Konkret wird die Hashfunktion bcrypt zum Hashen der Passwörter verwendet.

Um ein Passwort mit bcrypt zu Hashen muss zunächst ein Salt generiert werden. Neben dem Salt und dem gehastem Passwort wird bei bcrypt außerdem noch die bcyrpt Version und die Anzahl der Runden gespeichert. Über die Anzahl der Runden kann die Laufzeit die bcrypt benötigt um ein Passwort zu hashen beeinflusst werden.

Die verwendete Bibilithek ist jBCrypt von mindrot.org (org.mindrot.jbcrypt) in der Version 0.4. Sie verwendet intern die Java Klasse SecureRandom als CPRNG.

Auch zur Generierung der initalen (und temporären) Benutzer Passwörter beim Anlegen von Benutzer werden auf der Basis von SecureRandom generiert. Dafür zuständig ist die Klasse \textit{PasswordGenerator}. Die generierten Passwörter können die Zeichen a-z, A-Z, 0-9 und die Sonderzeichen \textit{!\%?\#-\_*+} enthalten.

\subsection{CSRF Tokens}
Play verwendet zum Generieren von CSRF Tokens die Java Klasse SecureRandom als CPRNG:

\begin{lstlisting}
  def generateToken: String = {
    val bytes = new Array[Byte](12)
    random.nextBytes(bytes)
    new String(Hex.encodeHex(bytes))
  }
\end{lstlisting}
[aus DefaultCSRFTokenSigner, Play 2.6.20]

Dieser Token wird allerdings nicht direkt verwendet. Stattdessen wird er mit einer nonce Konkateniert. Das Ergebniss wird anschließend mittels HMAC-SHA1 und einem privatem Schlüssel signiert. 

\begin{lstlisting}

  def signToken(token: String): String = {
    val nonce = clock.millis()
    val joined = nonce + "-" + token
    signer.sign(joined) + "-" + joined
  }
\end{lstlisting}

[aus DefaultCSRFTokenSigner, Play 2.6.20]

Das Format eines Play CSRF Token ist demnach
\begin{lstlisting}
<signature> - <nonce> - <token>
\end{lstlisting}


Das Token (aus Listing TODO) wird nur ein mal für eine Session generiert. Es bleibt innerhalb der Session immer gleich. Da das nonce auf der Systemzeit basiert, ändert sich diese jedoch mit jedem Request, was zu unterschiedlichen Signaturen führt.

Um ein Token zu prüfen wird erneut die Signatur über dem Token und dem Nounce gebildet und mit der empfangenen Signatur verglichen. Das modifizieren eines CSRF Tokens wird somit erkannt.



\subsection{Play Cookies}


\subsection{Schlüsselmanagement}
Für das System sind mehrer Schlüssel notwendig:

\begin{enumerate}
	\item Der private Schlüssel zum signieren der Play Cookies
	\item Der private Schlüssel für reCaptcha
	\item Der öffentlicht Schlüssel für reCaptcha
	\item Login Credentials für den E-Mail Server
	\item Login Credentials für die Datenbank
\end{enumerate}


\section{Eingabevalidierung}

\section{Ggf. architekturielle Gegenmassnahmen gegen bestimmte Angriffe}

\chapter{Implementierte Zusatzfunktionalitaet}

\textcolor{red}{Nutzer kann Gruppe selbst verlassen}
Nutzer koennen aktive Sessions betrachten und ggf. invalidieren (stichwort usability-relevantes securitykonzept)

\chapter{Verworfene Entwuerfe}

\printbibliography

% Can be used to add a list of acronyms with their description
%\glsaddall
%\deftranslation{to=German}{Acronyms}{Abkürzungsverzeichnis}
%\deftranslation{to=German}{Glossary}{Glossar}
\printacronyms[title=Abkürzungsverzeichnis,toctitle=Abkürzungsverzeichnis]
\printglossary[type=main]

%\addcontentsline{toc}{chapter}{\listfigurename}
\listoffigures      % Abbildungsverzeichnis

%s\addcontentsline{toc}{chapter}{\listtablename}
% \listoftables       % Tabellenverzeichnis

\end{document}
