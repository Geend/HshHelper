% arara: pdflatex: { shell: true, draft: true }
% arara: makeglossaries
% arara: biber
% arara: pdflatex: { shell: true, synctex: true }
% arara: pdflatex: { shell: true, synctex: true }

\documentclass[12pt,DIV14,BCOR10mm,a4paper,parskip=half-,headsepline,headinclude,english,ngerman,bibliography=totocnumbered]{scrreprt}

\usepackage{hshhelper_base}

%%%%%%%%%%%%%%%%%%%%%%%%%%%%%%%%%%%%%%%%%%%%%%%%%%%%%%%%%%%%%%%%%%%%%%%%%%
\begin{document}    % hier gehts los
  \thispagestyle{empty} % Titelseite
\includegraphics[width=0.2\textwidth]{Wortmarke_WI_schwarz}

   {  ~ \sffamily
  \vfill
  {\Huge\bfseries Bedrohungsanalyse}
  \bigskip

  {\Large
  Dennis Grabowski, Julius Zint, Philip Matesanz, Torben Voltmer \\[2ex]
  Masterprojekt \enquote{Entwicklung und Analyse einer sicheren \\Web-Anwendung} \\
  Wintersemester 18/19
 \\[5ex]
   \today }
}
 \vfill

  ~ \hfill
  \includegraphics[height=0.3\paperheight]{H_WI_Pantone1665}

\vspace*{-3cm}

\tableofcontents  % Inhaltsverzeichnis

\chapter{Zur Analyse verwendeten Methoden}
\section{Datenflussdiagramm}

Das folgende \gls{dfd} wurde mithilfe des Programms \enquote{OWASP Threat Dragon} \autocite{OWASP.ThreatDragon} erstellt.
Dieses Programm erlaubt leider keine Einfärbung in grün zur Kennzeichnung vertrauenswürdiger Prozesse oder Datenflüsse.
Alle schwarz gekennzeichneten Elemente sind daher als vertrauenswürdig zu betrachten.
Dieses DFD soll als Übersicht über alle vorhandenen Datenflüsse dienen, ohne dabei jeden Manager oder Controller separat aufzulisten, da für diese der Datenfluss der selbe wäre.

\begin{figure}[htbp]
  \hspace*{-2.5cm}
  \label{overview-dfd-pic}
  \includegraphics[width=1.25\linewidth]{resources/overview-dfd.jpg}
  \caption{Gesamtübersicht des Datenflusses in unserer Applikation}
\end{figure}

\subsection{Layered Entrypoints}

Die folgende Liste stellt alle vorhandenen Eintrittspunkte unserer Applikation in einer Baumstruktur vor.
Die Eintrittspunkte wurden für die bessere Übersicht nach \texttt{Controller} gruppiert, der für diesen Eintrittspunkt zuständig ist.
Zusätzlich werden die Parameter aufgelistet, die wir von dem Nutzer bei dem jeweiligen Eintrittspunkt entgegennehmen. Um diese kurz und übersichtlich zu halten, geben wir nicht die Session an, die ein Nutzer implizit mitschickt.
Sie dient hauptsächlich der Vollständigkeit und als Referenzliste, damit kein Eintrittspunkt übersehen werden kann.

\begin{enumerate}
  \item HTTP (Port 80)
  \begin{enumerate}
    \item \texttt{HomeController}
    \begin{enumerate}
     \item \texttt{/}
      \begin{enumerate}
        \item GET - \texttt{index()}
      \end{enumerate}
    \end{enumerate}

  \item \texttt{LoginController}
    \begin{enumerate}
     \item \texttt{/login}
      \begin{enumerate}
        \item GET - \texttt{showLoginForm()}
        \item POST - \texttt{login(username, password, recaptcha)}
      \end{enumerate}
      \item \texttt{/logout}
      \begin{enumerate}
        \item POST - \texttt{logout()}
      \end{enumerate}
      \item \texttt{/changePasswordAfterReset}
      \begin{enumerate}
        \item GET - \texttt{showChangePasswordAfterResetForm()}
        \item POST - \texttt{changePasswordAfterReset(username, currentPassword, password, passwordRepeat, recaptcha)}
      \end{enumerate}

      \item \texttt{/resetPassword}
      \begin{enumerate}
        \item GET - \texttt{showResetPasswordForm()}
        \item POST - \texttt{requestResetPassword(username, recaptcha)}
      \end{enumerate}

      \item \texttt{/resetPassword/:tokenId}
      \begin{enumerate}
        \item GET - \texttt{showResetPasswordWithTokenForm(tokenId)}
        \item POST - \texttt{resetPasswordWithToken(tokenId, newPassword)}
      \end{enumerate}
    \end{enumerate}

    \item \texttt{UserController}
    \begin{enumerate}
      \item \texttt{/users/all}
      \begin{enumerate}
        \item GET - \texttt{showUsers()}
      \end{enumerate}
      \item \texttt{/users/admins}
      \begin{enumerate}
        \item GET - \texttt{showAdminUsers()}
      \end{enumerate}
      \item \texttt{/users/create}
      \begin{enumerate}
        \item GET - \texttt{showCreateUserForm()}
        \item POST - \texttt{createUser(username, email, quotaLimit)}
      \end{enumerate}
      \item \texttt{/users/confirmDelete}
      \begin{enumerate}
        \item GET - \texttt{showConfirmDeleteForm(userId)}
      \end{enumerate}
      \item \texttt{/users/delete}
      \begin{enumerate}
        \item POST - \texttt{deleteUser(userId)}
      \end{enumerate}
      \item \texttt{/users/:userId/settings}
      \begin{enumerate}
        \item GET - \texttt{showUserAdminSettings(userId)}
      \end{enumerate}
      \item \texttt{/users/editQuota}
      \begin{enumerate}
        \item GET - \texttt{changeUserQuotaLimit(userId,
        newQuotaLimit)}
      \end{enumerate}
      \item \texttt{/users/deactivateTwoFactorAuth}
      \begin{enumerate}
        \item POST - \texttt{deactivateSpecificUserTwoFactorAuth(userId)}
      \end{enumerate}

      \item \texttt{/sessions}
      \begin{enumerate}
        \item GET - \texttt{showActiveUserSessions(userId)}
      \end{enumerate}
      \item \texttt{/sessions/delete}
      \begin{enumerate}
        \item POST - \texttt{deleteUserSession(sessionId)}
      \end{enumerate}
      \item \texttt{/settings}
      \begin{enumerate}
        \item POST - \texttt{showUserSettings()}
      \end{enumerate}
      \item \texttt{/settings/changePassword}
      \begin{enumerate}
        \item POST - \texttt{changeUserPassword(currentPassword, newPassword)}
      \end{enumerate}
      \item \texttt{/settings/sessionTimeout}
      \begin{enumerate}
        \item POST - \texttt{changeUserSessionTimeout(newSessionTimeout)}
      \end{enumerate}

      \item \texttt{/settings/confirmActivateTwoFactorAuth}
      \begin{enumerate}
        \item GET - \texttt{show2FactorAuthConfirmationForm(generatedTwoFactorSecret, QRCodeImage)}
      \end{enumerate}
      \item \texttt{/settings/activateTwoFactorAuth}
      \begin{enumerate}
        \item POST - \texttt{activateTwoFactorAuth(twoFactorSecret, activationToken)}
      \end{enumerate}
      \item \texttt{/settings/deactivateTwoFactorAuth}
      \begin{enumerate}
        \item POST - \texttt{deactivateTwoFactorAuth(userId)}
      \end{enumerate}
    \end{enumerate}

    \item \texttt{GroupController}
    \begin{enumerate}
      \item \texttt{/groups/all}
      \begin{enumerate}
        \item GET - \texttt{showAllGroups()}
      \end{enumerate}
      \item \texttt{/groups/membership}
      \begin{enumerate}
        \item GET - \texttt{showOwnMemberships()}
      \end{enumerate}
      \item \texttt{/groups/own}
      \begin{enumerate}
        \item GET - \texttt{showOwnGroups()}
      \end{enumerate}
      \item \texttt{/groups/delete}
      \begin{enumerate}
        \item POST - \texttt{deleteGroup(groupId)}
      \end{enumerate}
      \item \texttt{/groups/confirmDelete}
      \begin{enumerate}
        \item POST - \texttt{confirmDelete(groupId)}
      \end{enumerate}
      \item \texttt{/groups/create}
      \begin{enumerate}
        \item GET - \texttt{showCreateGroupForm()}
        \item POST - \texttt{createGroup(groupName)}
      \end{enumerate}

      \item \texttt{/groups/:groupId}
      \begin{enumerate}
        \item GET - \texttt{showGroup(groupId)}
      \end{enumerate}
      \item \texttt{/groups/:groupId/files}
      \begin{enumerate}
        \item GET - \texttt{showGroupFiles(groupId)}
      \end{enumerate}
      \item \texttt{/groups/:groupId/members}
      \begin{enumerate}
        \item GET - \texttt{showGroupMembers(groupId)}
      \end{enumerate}
      \item \texttt{/groups/:groupId/members/remove}
      \begin{enumerate}
        \item POST - \texttt{removeGroupMember(groupId, userId)}
      \end{enumerate}
      \item \texttt{/groups/:groupId/members/add}
      \begin{enumerate}
        \item GET - \texttt{showAddMemberForm(groupId)}
        \item POST - \texttt{addGroupMember(groupId, userId)}
      \end{enumerate}
    \end{enumerate}

    \item \texttt{FileController}
    \begin{enumerate}

      \item \texttt{/files/own}
      \begin{enumerate}
        \item GET - \texttt{showOwnFiles()}
      \end{enumerate}
      \item \texttt{/files/shared}
      \begin{enumerate}
        \item GET - \texttt{showSharedFiles()}
      \end{enumerate}
      \item \texttt{/files/thirdParty}
      \begin{enumerate}
        \item GET - \texttt{showThirdPartyFiles()}
      \end{enumerate}
      \item \texttt{/files/delete}
      \begin{enumerate}
        \item POST - \texttt{deleteFile(fileId)}
      \end{enumerate}

      \item \texttt{/files/upload}
      \begin{enumerate}
        \item GET - \texttt{showUploadFileForm()}
        \item POST - \texttt{uploadFile(fileName, fileContent, comment, groupPermissions, userPermissions)}
      \end{enumerate}

      \item \texttt{/files/uploadToGroup/:groupId}
      \begin{enumerate}
        \item GET - \texttt{showUploadFileToGroupForm(groupId)}
      \end{enumerate}
      \item \texttt{/files/quota}
      \begin{enumerate}
        \item GET - \texttt{showQuotaUsage()}
      \end{enumerate}
      \item \texttt{/files/:fileId}
      \begin{enumerate}
        \item GET - \texttt{showFile(fileId)}
      \end{enumerate}
      \item \texttt{/files/:fileId/download}
      \begin{enumerate}
        \item GET - \texttt{downloadFile(fileId)}
      \end{enumerate}
      \item \texttt{/files/editComment}
      \begin{enumerate}
        \item POST - \texttt{editFileComment(fileId, comment)}
      \end{enumerate}
      \item \texttt{/files/editContent}
      \begin{enumerate}
        \item POST - \texttt{editFileContent(fileId, newFileContent)}
      \end{enumerate}
      \item \texttt{/files/search}
      \begin{enumerate}
        \item POST - \texttt{searchFiles(query)}
      \end{enumerate}
    \end{enumerate}

    \item \texttt{PermissionController}
    \begin{enumerate}

      \item \texttt{/permissions/editUserPermission}
      \begin{enumerate}
        \item POST - \texttt{showEditUserPermissionForm(userPermissionsId)}
      \end{enumerate}
      \item \texttt{/permissions/editUserPermission/submit}
      \begin{enumerate}
        \item POST - \texttt{editUserPermission(userPermissionId, permissionLevel, returnUrl)}
      \end{enumerate}
      \item \texttt{/permissions/editGroupPermission/}
      \begin{enumerate}
        \item POST - \texttt{showEditGroupPermissionForm(userPermissionId, returnUrl)}
      \end{enumerate}
      \item \texttt{/permissions/editGroupPermission/submit}
      \begin{enumerate}
        \item POST - \texttt{editGroupPermission(groupPermissionId, permissionLevel, returnUrl)}
      \end{enumerate}

      \item \texttt{/permissions/createUserPermission/:fileId}
      \begin{enumerate}
        \item GET - \texttt{showCreateUserPermission(fileId)}
      \end{enumerate}
      \item \texttt{/permissions/createUserPermission}
      \begin{enumerate}
        \item POST - \texttt{createUserPermission(fileId, userId, permissionLevel)}
      \end{enumerate}
      \item \texttt{/permissions/createGroupPermission/:fileId}
      \begin{enumerate}
        \item GET - \texttt{showCreateGroupPermission(fileId)}
      \end{enumerate}
      \item \texttt{/permissions/createGroupPermission/}
      \begin{enumerate}
        \item POST - \texttt{createGroupPermission(fileId, groupId, permissionLevel)}
      \end{enumerate}

      \item \texttt{/permissions/deleteGroupPermission}
      \begin{enumerate}
        \item POST - \texttt{deleteGroupPermission(groupPermissionId, returnUrl)}
      \end{enumerate}
      \item \texttt{/permissions/deleteUserPermission}
      \begin{enumerate}
        \item POST - \texttt{deleteUserPermission(userPermissionId, returnUrl)}
      \end{enumerate}
    \end{enumerate}

    \item \texttt{ErrorController}
    \begin{enumerate}
     \item \texttt{/forbidden}
      \begin{enumerate}
        \item GET - \texttt{showForbiddenMessage()}
      \end{enumerate}
      \item \texttt{/badrequest}
      \begin{enumerate}
        \item GET - \texttt{showBadRequestMessage()}
      \end{enumerate}
    \end{enumerate}

    \item \texttt{NetServiceController}
    \begin{enumerate}
      \item \texttt{/netservices/all}
        \begin{enumerate}
          \item GET - \texttt{showAllNetServices()}
        \end{enumerate}
      \item \texttt{/netservices/create}
        \begin{enumerate}
          \item GET - \texttt{showAddNetServiceForm()}
          \item POST - \texttt{createNetService(netServiceName, netServiceURL)}
        \end{enumerate}
      \item \texttt{/netservices/delete}
        \begin{enumerate}
          \item POST - \texttt{deleteNetService(netServiceId)}
        \end{enumerate}
      \item \texttt{/netservices/confirmDelete}
        \begin{enumerate}
          \item POST - \texttt{showDeleteNetServiceConfirmation()}
        \end{enumerate}
        \item \texttt{/netservices/edit/}
        \begin{enumerate}
          \item POST - \texttt{editNetService(netServiceId, netServiceName, netServiceURL)}
        \end{enumerate}
      \item \texttt{/netservices/edit/:netServiceId}
        \begin{enumerate}
          \item GET - \texttt{showEditNetService(netServiceId)}
        \end{enumerate}
      \item \texttt{/netservices/addparameter}
        \begin{enumerate}
          \item POST - \texttt{addNetServiceParameter(netServiceId, netServiceName, netServiceParameterType, defaultValue)}
        \end{enumerate}
      \item \texttt{/netservices/removeparameter}
        \begin{enumerate}
          \item POST - \texttt{removeNetServiceParameter(netServiceId, netServiceParameterId)}
        \end{enumerate}


     \item \texttt{/credentials}
      \begin{enumerate}
        \item GET - \texttt{showUserNetServiceCredentials()}
      \end{enumerate}
      \item \texttt{/credentials/:credentialId}
      \begin{enumerate}
        \item GET - \texttt{decryptNetServiceCredential(credentialId)}
      \end{enumerate}
      \item \texttt{/credentials/create}
      \begin{enumerate}
        \item GET - \texttt{showCreateNetServiceCredentialForm()}
        \item POST - \texttt{createNetServiceCredential(netServiceId, netServiceCredentialUsername, netServiceCredentialPassword)}
      \end{enumerate}
      \item \texttt{/credentials/delete}
      \begin{enumerate}
        \item GET - \texttt{deleteNetServiceCredential(netServiceCredentialId)}
      \end{enumerate}
    \end{enumerate}

  \item \texttt{/assets/*file}
    \begin{enumerate}
      \item GET - \texttt{Assets.at(path="/public", file)}
    \end{enumerate}
  \end{enumerate}
  \item SMTP (Port 587)
  \begin{enumerate}
    \item Versand der E-Mail für den Passwort-Reset
  \end{enumerate}
\end{enumerate}

\chapter{Bedrohungen}
\section{Gefundene Bedrohungen}

\subsection{Globale Bedrohungen}

Unser Schemata zur Bewertung des Risikos einer Bedrohung haben wir teils an die \enquote{OWASP Risk Rating Methodology} angelehnt \footnote{Siehe \url{https://www.owasp.org/index.php/OWASP_Risk_Rating_Methodology}}. Wir verwenden die selbe Unterteilung von Auswirkung und Wahrscheinlichkeit in 3 Stufen, aber bewerten das abschließende Risiko in 4 Stufen \enquote{Note}, \enquote{Gering}, \enquote{Mittel}, \enquote{Hoch}.

\begin{itemize}

  \hypertarget{threat1}{}
  \item \textbf{T1: Spoofing des Servers}
  \begin{itemize}
  \item Zweck:
  	\begin{itemize}
  		\item Ausspähen von Anmeldedaten der Benutzer
  		\item Ausspähen von Dateien, die ein Benutzer beim HsH-Helper hochladen möchte
  	\end{itemize}
  \item Möglichkeiten:
  	\begin{itemize}
  		\item IP-Spoofing
  		\item DNS manipulieren
  	\end{itemize}
  \item Risiko: Mittel
  \item Mögliche Gegenmaßnahmen: Authentifizierung des Servers durch Zertifikate (HTTPS mit HSTS)
  \end{itemize}

  \hypertarget{threat2}{}
  \item \textbf{T2: Eavesdropping}
  \begin{itemize}
  \item Zweck:
  	\begin{itemize}
  		\item Primär Ausspähen von Anmeldedaten der Benutzer
  		\item Grundsätzlich sind allen Daten (Gruppennamen etc.) Ziel
  		%\item Ausspähen von Dateien, die ein Benutzer bei HsH-Helfer hochlädt
  		%\item Ausspähen von Dateien, die ein Benutzer bei HsH-Helfer runterlädt
  	\end{itemize}
  \item Möglichkeiten: Man-in-the-Middle-Angriff
  \item Risiko: Mittel
  \item Mögliche Gegenmaßnahmen: Verschlüsselung durch HTTPS mit HSTS oder SecureCookie
  \end{itemize}

  \hypertarget{threat3}{}
  \item \textbf{T3: Replayattacken}
  \begin{itemize}
  \item Zweck: Mehrfach-Ausführung geschützter Aktionen
  \item Möglichkeiten (Bezogen auf unsere Anwendung): Keine
  \item Risiko: Note
  \item Mögliche Gegenmaßnahmen: HTTPS %/ CSRF-Token
  \end{itemize}

  \hypertarget{threat4}{}
  \item \textbf{T4: Tampering}
  \begin{itemize}
  \item Zweck:
  	\begin{itemize}
  		\item Ausführung von nicht intendierten Aktionen (z.B. Benutzer-Löschen abfangen und Ziel-User-Id manipulieren)
  		\item Server-Response manipulieren (z.B. bösartiges JavaScript einbauen)
	\end{itemize}
  \item Möglichkeiten: Leitung splicen und Man-In-The-Middle durchführen
  \item Risiko: Mittel
  \item Mögliche Gegenmaßnahmen: HTTPS
  \end{itemize}

  \hypertarget{threat5}{}
  \item \textbf{T5: Verwendung eines gefälschten,von einer anerkannten Zertifizierungsstelle signierten Zertifikates}
  \begin{itemize}
  \item Zweck:
  	\begin{itemize}
  		\item Spoofing des Servers
  		\item Man-in-the-Middle-Angriffe
   \end{itemize}
  \item Möglichkeiten: Kontrolliert man eine anerkannten Zertifizierungsstelle, können gefälschte Zertifikate für die HsH-Helper Domain ausgestellt werden. Für HsH-Helper Benutzer ist das nicht sofort einsehbar\footnote{Bereits geschehen durch türkischen Geheimdienst: \url{https://arstechnica.com/information-technology/2013/01/turkish-government-agency-spoofed-google-certificate-accidentally/}}.
  \item Risiko: Gering
  \item Mögliche Gegenmaßnahmen: HTTP Public Key Pinning (HPKP)
  \end{itemize}

  \hypertarget{threat6}{}
  \item \textbf{T6: XSS (Cross Site Scripting)}
  \begin{itemize}
  \item Zweck: Ausführung von JavaScript im HsH-Helper Kontext eines Benutzers
  \item Möglichkeiten: Eingaben, die ausgegeben werden, ohne Eingabevalidierung oder Ausgabekodierung auszuführen
  \item Risiko: Hoch
  \item Mögliche Gegenmaßnahmen:
	\begin{itemize}
		\item Eingabevalidierung (Kleinbuchstaben und Zahlen, valide E-Mail-Adressen)
		\item X-XSS-Protection (vgl. \ref{x:xss:protection})
		\item Content Security Policy (CSP) des Browsers aktivieren
		\item Alte Browser beziehungsweise Browser, die CSP nicht können, aussperren
		\item Ausgabekodierung
	\end{itemize}
  \end{itemize}

  \hypertarget{threat7}{}
  \item \textbf{T7: SQL Injection}
  \begin{itemize}
  \item Zweck: Unzulässiges Lesen und Schreiben von Datenbankinhalten
  \item Möglichkeiten: Inline-SQL Queries ohne Seperation of Data and Code
  \item Risiko: Hoch
  \item Mögliche Gegenmaßnahmen:
  \begin{itemize}
    \item Die H2 Einstellung \texttt{ALLOW\_LITERALS=NONE} verwenden
    \item Nur PreparedStatements benutzen
    \item Eingabevalidierung
  \end{itemize}
  \end{itemize}

  \hypertarget{threat8}{}
  \item \textbf{T8: Offline Password Brute Forcing}
  \begin{itemize}
  \item Zweck: Durch SQL Injection erbeutete Hashes in Klartext verwandeln
  \item Möglichkeiten: Rainbow Tables
  \item Risiko: Gering
  \item Mögliche Gegenmaßnahmen:
  \begin{itemize}
    \item Passwörter nur gehashed mit Salt speichern
    \item Passwörter zusätzlich mit Pepper hashen
  \end{itemize}
  \end{itemize}

  \hypertarget{threat9}{}
  \item \textbf{T9: Cross-Site-Request-Forgery (CSRF)}
  \begin{itemize}
  \item Zweck: Benutzer dazu bringen Aktionen auszuführen, die sie nicht intendiert haben
  \item Möglichkeiten:
  \begin{itemize}
          \item Einen fremden Benutzer dazu bringen, sich selbst auszuloggen
          \item Einem Administrator einen Request unterjubeln, durch welchen er einen Nutzer erstellt
  \end{itemize}
  \item Risiko: Hoch
  \item Mögliche Gegenmaßnahmen: CSRF-Tokens bei Requests prüfen
  \end{itemize}

  \hypertarget{threat10}{}
  \item \textbf{T10: Session fälschen (Bestimmte oder Irgendeine)}
  \begin{itemize}
  \item Zweck: Client spoofen um Aktionen auszuführen, für die der gespoofte Client autorisiert ist
  \item Möglichkeiten: Cookie fälschen
  \item Risiko: Hoch
  \item Mögliche Gegenmaßnahmen:
  \begin{itemize}
    \item Signieren des Cookies durch privaten Schlüssel,
    \item Verwendung von UUID,
    \item Bindung der Session an IP-Adresse,
    \item Begrenzung der Lebenszeiten
  \end{itemize}
  \end{itemize}

  \hypertarget{threat11}{}
  \item \textbf{T11: Abstreitbarkeit illegaler Handlungen}
  \begin{itemize}
  \item Zweck: Anderen Nutzer in rechtliche Schwierigkeiten bringen
  \item Möglichkeiten: Eintragen illegaler Namen bei Benutzernamen oder Gruppenname
  \item Risiko: Gering
  \item Mögliche Gegenmaßnahmen: Logging von IP + Zeit, Action, User
  \end{itemize}

  \hypertarget{threat12}{}
  \item \textbf{T12: Nutzer kann eine ihm unerlaubte Operation durchführen}
  \begin{itemize}
  \item Zweck: Beispielsweise um einen Denial-of-Service-Angriff auf einen Nutzer durchzuführen
  \item Möglichkeiten: Beispielsweise Nutzer löscht andere Nutzer
  \item Risiko: Mittelmäßig - Hoch
  \item Mögliche Gegenmaßnahmen: Autorisierung erzwingen und prüfen bei jeder Operation
  \end{itemize}

  \hypertarget{threat13}{}
  \item \textbf{T13: Captchas durch menschliche Akteure lösen lassen}
  \begin{itemize}
  \item Zweck: Brute-Forcing ermöglichen
  \item Möglichkeiten: 2captcha.com bietet diesen Service für 2.99\$ für 1000 reCAPTCHAs
  \item Risiko: Gering
  \item Mögliche Gegenmaßnahmen: Zwei-Faktor-Authentisierung
  \end{itemize}

  \hypertarget{threat14}{}
  \item \textbf{T14: Distributed-Denial-of-Service}
  \begin{itemize}
  \item Zweck: Verfügbarkeit der Anwendung beeinträchtigen
  \item Möglichkeiten: Botnetzwerk oder Booter-Service verwenden
  \item Risiko: Mittel
  \item Mögliche Gegenmaßnahmen: Cloudflare/Incapsula/Akamai Secure Shield verwenden
  \end{itemize}

  %\item \textcolor{red}{Elevation of privilege -> Frage an Peine. ?}
\end{itemize}

\subsection{Domänenspezifische Bedrohungen}

\subsubsection{Mögliche Angriffe auf den Loginprozess}

\begin{itemize}

  \hypertarget{threat15}{}
  \item \textbf{T15: Überlanges Passwort beim Login eingeben}
  \begin{itemize}
  \item Zweck: DOS der Applikation
  \item Möglichkeiten: Durch überlange Passwörter eine ressourcenintensive Hash-Operation in Gang setzen
  \item Risiko: Hoch
  \item Mögliche Gegenmaßnahmen:
  \begin{itemize}
  \item Passwort-Länge mittels Eingabevalidierung beschränken
  \item Verwendung von bcrypt (lässt nur 72 Zeichen lange Passwörter zu, Laufzeit ist für alle Eingabelängen konstant)
  \end{itemize}
\end{itemize}

  \hypertarget{threat35}{}
  \item \textbf{T35: Hash-Algorithmus als DOS Angriffsvektor missbrauchen}
  \begin{itemize}
  \item Zweck: DOS der Applikation
  \item Möglichkeiten: Wenige Requests über eine ressourcenintensive Hash-Funktion amplifizieren, um eine hohe CPU-Auslastung des Servers zu erreichen.
  \item Risiko: Hoch
  \item Mögliche Gegenmaßnahmen:
  \begin{itemize}
  \item Schnellen Hash-Algorithmus verwenden (z.B. SHA512)
  \item Captcha vor Hash-Berechnung schalten.
  \item Zugriffe protokollieren und bei Attacke den Angreifer blockieren.
  \end{itemize}
\end{itemize}

  \hypertarget{threat16}{}
  \item \textbf{T16: Timing-Angriff auf Loginverhalten}
  \begin{itemize}
  \item Zweck: Herausfinden, ob ein Benutzer mit einem bestimmten Username existiert
  \item Möglichkeiten: Durch Messen der Antwortszeiten des Servers können Rückschlüsse darüber gezogen werden, welcher Code ausgeführt wurde (zB Hashing/DB-Zugriff)
  \item Risiko: Mittel
  \item Mögliche Gegenmaßnahmen: Loginprozess für jeden Fall gleich ablaufen lassen
  \end{itemize}

  \hypertarget{threat17}{}
  \item \textbf{T17: Online Brute-Forcing}
  \begin{itemize}
  \item Zweck: Unautorisierter Zugriff auf ein fremdes Benutzerkonto
  \item Möglichkeiten: Beispielsweise durch Wörterbuchattacken
  \item Risiko: Hoch
  \item Mögliche Gegenmaßnahmen:
  \begin{itemize}
      \item Login nur möglich, wenn Captcha gelöst wird
      \item IP Sperre bei auffälligem Verhalten
      \item Schwache Passwörter verbieten
      \item Zwei-Faktor-Authentisierung
    \end{itemize}
  \end{itemize}
\end{itemize}

\subsubsection{Mögliche Angriffe auf den \enquote{Passwort zurücksetzen lassen}-Prozess}

\begin{itemize}
  \hypertarget{threat18}{}
  \item \textbf{T18: Nutzer permanent aus der Applikation ausschließen}
  \begin{itemize}
  \item Zweck: Denial-of-Service-Angriff auf einen individuellen Nutzer
  \item Möglichkeiten: Wiederholtes Absenden des Request mit richtigen Nutzernamen generiert bei jedem Request ein neues temporäres Passwort. Selbst wenn der Benutzer das temporäre Passwort ändert, wird er es nicht schaffen sich anzumelden, da dann bereits die Generierung ein neues temporäres Passwort vom Angreifer erzwungen wurde.
  \item Risiko: Hoch
  \item Mögliche Gegenmaßnahmen:
  \begin{itemize}
  \item Einbau eines Captchas
  \item Dem Benutzer einen Link schicken, der geöffnet werden muss, um den \enquote{Passwort nach Zurücksetzung ändern}-Prozess anzustoßen. Der Link muss einen geheimen Token beinhalten, der nur HsH-Helper und dem Empfänger der E-Mail bekannt ist. Das temporäre Passwort wird erst generiert, sobald auf den Link geklickt wird.
  \end{itemize}
\end{itemize}

  \hypertarget{threat19}{}
  \item \textbf{T19: Timing-Angriff auf Verhalten des Requests}
  \begin{itemize}
  \item Zweck: Herausfinden, ob ein Benutzer mit einem bestimmten Username existiert
  \item Möglichkeiten: Durch Messen der Antwortszeiten des Servers können Rück\-schlüsse darüber gezogen werden, welcher Code ausgeführt wurde (E-Mail-Versand)
  \item Risiko: Mittel
  \item Mögliche Gegenmaßnahmen:
  \begin{itemize}
    \item \enquote{Passwort zurücksetzen lassen}-Prozess in konstanter Länge ablaufen lassen, ggf. durch Absenden einer E-Mail an eine Dummy-E-Mail-Adresse
    \item E-Mail-Versand asynchron durchführen
    \end{itemize}
  \end{itemize}

  \hypertarget{threat20}{}
  \item \textbf{T20: E-Mail Postfach des Nutzers fluten}
  \begin{itemize}
  \item Zweck: Verbrauch der vorhandenen Resourcen seines E-Mail-Kontos
  \item Möglichkeiten: Wiederholtes Absenden des Formulars durch Angabe seines Nutzernamens
  \item Risiko: Hoch
  \item Mögliche Gegenmaßnahmen: Nutzer muss Captcha lösen, bevor Prozess angestoßen werden darf
  \end{itemize}
  
  \hypertarget{threat38}{}
  \item \textbf{T38: Aushebeln der Anforderung zur Offline-Übergabe des Passworts}
  \begin{itemize}
  \item Zweck: Zugriff auf den Benutzeraccount erhalten, bevor er offiziell ausgehändigt wurde.
  \item Möglichkeiten: Administrator legt Benutzeraccount vorab an. Nutzer kennt die zugehörige E-Mail, da er sie der Hochschule mitgeteilt hat. Nutzer kann Passwort-Reset durchführen und das Passwort ändern. Er erhält so Zugriff den Account bevor ihm das Passwort vom Administrator mitgeteilt wurde.
  \item Risiko: Low
  \item Mögliche Gegenmaßnahmen:
  \begin{itemize}
	  \item Passwort-Reset erst erlauben, wenn Nutzer initiales Passwort geändert hat
  \end{itemize}
\end{itemize}
\end{itemize}

\subsubsection{Mögliche Angriffe auf den \enquote{Passwort nach Zurücksetzung ändern}-Prozess}

\begin{itemize}

  \hypertarget{threat36}{}
  \item \textbf{T36: Passwort-Reset-Token erraten}
  \begin{itemize}
  \item Zweck: Passwort eines Nutzers ändern, ohne Zugriff auf sein E-Mail Konto und derzeitiges Passwort zu haben.
  \item Möglichkeiten: Brute-Forcing des entsprechenden Tokens.
  \item Risiko: Hoch
  \item Mögliche Gegenmaßnahmen:
  \begin{itemize}
  \item Token mit ausreichender Länge zufällig generieren
  \item Token an IP-Addresse koppeln
  \item Gültigkeit des Token zeitlich begrenzen
  \end{itemize}
\end{itemize}

  \hypertarget{threat37}{}
  \item \textbf{T37: Umgehung des zweiten Faktors}
  \begin{itemize}
  \item Zweck: Löschen der Netzdienst-Zugangsdaten.
  \item Möglichkeiten: Angreifer erlangt Zugriff auf den Passwort-Zurücksetzen-Link bzw. das E-Mail Konto des Opfers und kann das Passwort ändern. Im gleichen Prozess werden die Zugangsdaten des Opfers gelöscht. Der Angreifer hat mangels zweiten Faktor kein Zugriff auf den Account des Opfers, konnte jedoch  die Zugangsdaten löschen.
  \item Risiko: Gering
  \item Mögliche Gegenmaßnahmen:
  \begin{itemize}
  	\item Zweiten Faktor auch beim Passwort-Zurücksetzen valdieren
  	\item Zugangsdaten nicht automatisch beim Zurücksetzen löschen
  \end{itemize}
\end{itemize}

\end{itemize}


\subsection{Mögliche Angriffe auf den \enquote{Datei hochladen}-Prozess}

\begin{itemize}
  \hypertarget{threat21}{}
  \item \textbf{T21: Dateinamen, Kommmentarfeld oder Gruppennamen zur Dateispeicherung missbrauchen}
  \begin{itemize}
  \item Zweck: Speicherplatzlimit umgehen, Speicherverbrauch des Servers hochtreiben
  \item Möglichkeiten: Dateiinhalt in Kommentarfeld oder Dateinamen schreiben, gegebenfalls sogar kodiert
  \item Risiko: Mittel
  \item Mögliche Gegenmaßnahmen:
  \begin{itemize}
    \item Kommentarfeldlänge und Dateinamenlänge zählen ins Speicherplatzlimit des Nutzers
    \item Maximallänge eines Gruppennamen begrenzen
  \end{itemize}
  \end{itemize}
\end{itemize}

\subsection{Mögliche Angriffe auf den \enquote{Dateien anzeigen lassen}-Prozess}

\begin{itemize}
  \hypertarget{threat22}{}
  \item \textbf{T22: Irreführung des Nutzers durch ähnliche Dateinamen}
  \begin{itemize}
  \item Zweck: Nutzer eine bösartige Datei unterjubeln
  \item Möglichkeiten: Datei mit gleichem Dateinamen hochladen, wie besagter Nutzer bereits hochgeladen hat und ihm Lese- und Schreibrechte geben
  \item Risiko: Hoch
  \item Mögliche Gegenmaßnahmen: Hervorhebung der Nutzer-eigenen Datei, beispielsweise durch Farben oder Symbolen
  \end{itemize}
\end{itemize}

\subsection{Mögliche Angriffe auf den \enquote{Auf Datei zugreifen}-Prozess}

\begin{itemize}
  \hypertarget{threat23}{}
  \item \textbf{T23: Dateien hochladen, die vom Browser interpretiert werden}
  \begin{itemize}
  \item Zweck: Code-Injection, Cookies klauen
  \item Möglichkeiten: Cookies klauen durch JavaScript-Injektion
  \item Risiko: Kritisch
  \item Mögliche Gegenmaßnahmen:
    \begin{itemize}
      \item HTTP-Header \texttt{Content-Type} auf \texttt{application/octet-stream} setzen
      \item User-generierte Inhalte über eine andere Domain ausliefern
      \item Dateien beim Download zippen
    \end{itemize}
  \end{itemize}

  \hypertarget{threat24}{}
  \item \textbf{T24: Dateiname erlaubt HTTP Response Splitting}
  \begin{itemize}
  \item Zweck: Schadcode injizieren
  \item Möglichkeiten: Doppelte Newlines im Dateinamen
  \item Risiko: Hoch
  \item Mögliche Gegenmaßnahmen: Eingabevalidierung, damit Newlines nicht mehr erlaubt sind
  \end{itemize}
\end{itemize}

\subsubsection{Mögliche Angriffe auf Zusatzfunktionalität: Datei-Berechtigungen editieren}

\begin{itemize}

  \hypertarget{threat25}{}
  \item \textbf{T25: Open Redirect}
  \begin{itemize}
    \item Zweck: Benutzer auf eine Angriffseite umleiten
    \item Möglichkeiten: Eigene URL als Redirect-Parameter angeben
    \item Risiko: Mittel
    \item Mögliche Gegenmaßnahmen:
      \begin{itemize}
      	\item Zugrundeliegende Funktionalität nicht per GET sondern per POST anbieten, das vom Play CSRF-Schutz gesichert wird
      	\item URL-Eingabe durch ein Regex prüfen
      	\item URL-Eingabe durch eine Whitelist prüfen
      \end{itemize}
  \end{itemize}
\end{itemize}

\subsubsection{Mögliche Angriffe auf das Logging}

\begin{itemize}

  \hypertarget{threat26}{}
  \item \textbf{T26: Log Poisoning}
  \begin{itemize}
  \item Zweck: Log-Einträge für real nicht stattgefundene Events erstellen / Anderem Nutzer etwas anhängen
  \item Möglichkeiten: User-Input in Logs mit Newline versehen
  \item Risiko: Mittel
  \item Mögliche Gegenmaßnahmen:
  \begin{itemize}
    \item Ausgabekodierung von User-Input in Logdatei z.B. Base64
    \item Eingabevalidierung, z.B.: Username, der durch Regex gefiltert wird
    \item Eigenen Logger implementieren, der potentiell gefährliche Zeichen filtert
  \end{itemize}
  \end{itemize}

  \hypertarget{threat27}{}
  \item \textbf{T27: Log Code Injection}
  \begin{itemize}
  \item Zweck: Programme angreifen, welche die Log parsen/öffnen
  \item Möglichkeiten: Schadcode in User-Input eingeben, der in der Log-Datei landet
  \item Risiko: Low
  \item Mögliche Gegenmaßnahmen:
    \begin{itemize}
      \item Nur User-Input loggen, in welchen kein Schadcode versteckt werden kann
    \end{itemize}
  \end{itemize}

  \hypertarget{threat28}{}
  \item \textbf{T28: Speicherverbrauch durch Spammen von Requests}
  \begin{itemize}
  \item Zweck: Speicher des Servers völlig mit Logs belegen (Denial of Service)
  \item Möglichkeiten: In Endlosschleife Requests an den Server senden
  \item Risiko: Mittel
  \item Mögliche Gegenmaßnahmen: Log-Rotation verwenden, um ältere Logs auf externen Speicher persistieren
  \end{itemize}

  \hypertarget{threat29}{}
  \item \textbf{T29: Log-Deletion durch Spammen von Requests}
  \begin{itemize}
  \item Zweck: Log-Rotation triggern, um eine Beweisvernichtung auszuführen
  \item Möglichkeiten: In Endlosschleife Requests an den Server senden
  \item Risiko: Mittel
  \item Mögliche Gegenmaßnahmen:
    \begin{itemize}
      \item Log-Rotation ohne automatische Löschung benutzen
  	\end{itemize}
  \end{itemize}

  \hypertarget{threat30}{}
  \item \textbf{T30: Virensignatur in User-Input}
  \begin{itemize}
  \item Zweck: Denial of Service / Logging blockieren
  \item Möglichkeiten: Eine Signatur in zuloggenden User-Input einbauen, die von einem Virenscanner erkannt wird, wodurch dieser Zugriff auf die Log-Datei blockiert
  \item Risiko: Low
  \item Mögliche Gegenmaßnahmen:
  \begin{itemize}
  	\item Keinen Virenscanner auf Applikationsserver verwenden
  	\item Anwendung in VM ohne Virenscanner laufen lassen
  	\item Log-Folder auf Whitelist des Virenscanner setzen
  	\item Nur validierten User-Input loggen; beispielsweise Username, der durch ein Regex gefiltert wird
  \end{itemize}
  \end{itemize}
\end{itemize}

\subsubsection{Mögliche Angriffe auf Zusatzfunktionalität: Anzeige, wer Dateiinhalt geschrieben hat}


\begin{itemize}

  \hypertarget{threat31}{}
  \item \textbf{T31: CSS Injection}
  \begin{itemize}
  \item Zweck: Einfärbewarnung deaktivieren/manipulieren
  \item Möglichkeiten: Wie XSS, jedoch wird anstelle von Javascript Inline-CSS injiziert, das die CSS-Klasse überschreibt, welche für die Rotfärbung von Dateien verantwortlich ist
  \item Risiko: Mittel
  \item Mögliche Gegenmaßnahmen:
  \begin{itemize}
	  \item Ausgabekodierung
	  \item Content-Security-Policy, die Inline-CSS verbietet
  \end{itemize}
\end{itemize}
\end{itemize}

\subsubsection{Mögliche Angriffe auf Zusatzfunktionalität: Nutzer kann Session-Timeout einstellen}

\begin{itemize}

  \hypertarget{threat32}{}
  \item \textbf{T32: Überlanges Session-Timeout einstellen}
  \begin{itemize}
    \item Zweck: Umgehen des intendierten Schutzmechanismus (Session-Timeout)
    \item Möglichkeiten: Einen extrem hohen Wert wählen
    \item Risiko: Mittel
    \item Mögliche Gegenmaßnahmen:
      \begin{itemize}
      \item Einen Maximalwert definieren
      \end{itemize}
  \end{itemize}

  \hypertarget{threat33}{}
  \item \textbf{T33: Integer Overflow}
  \begin{itemize}
    \item Zweck: Umgehen des intendierten Schutzmechanismus (Session-Timeout)
    \item Möglichkeiten: Einen extrem hohen Wert wählen
    \item Risiko: Mittel
    \item Mögliche Gegenmaßnahmen:
      \begin{itemize}
      \item Einen Maximalwert definieren
      \item Arithmetik verwenden, die nicht anfällig für Integer-Overflows ist
      \end{itemize}
  \end{itemize}

  \hypertarget{threat34}{}
  \item \textbf{T34: Negatives Session Timeout}
  \begin{itemize}
    \item Zweck: Umgehen des intendierten Schutzmechanismus (Session-Timeout)
    \item Möglichkeiten: Einen sehr geringen (negativen) Wert wählen
    \item Risiko: Mittel
    \item Mögliche Gegenmaßnahmen:
      \begin{itemize}
      \item Einen Minimalwert definieren
      \end{itemize}
  \end{itemize}
\end{itemize}

\subsubsection{Mögliche Angriffe auf Zusatzfunktionalität: Zwei-Faktor-Authentisierung}
\begin{itemize}

  \hypertarget{threat39}{}
  \item \textbf{T39: Secret für Zwei-Faktor-Authentisierung ist erratbar}
  \begin{itemize}
    \item Zweck: Umgehen des intendierten Schutzmechanismus (Zwei-Faktor-Authentisierung)
    \item Möglichkeiten: Secret basiert nicht auf kryptografisch sicheren Zufallszahlen. Angreifer kann es erraten und selbst in Authentifizierungs-App eingeben. 
    \item Risiko: Mittel
    \item Mögliche Gegenmaßnahmen:
      \begin{itemize}
      \item SecureRandom zur Erstellung der Secrets verwenden
      \item Ausreichend langes Secret erstellen
      \end{itemize}
  \end{itemize}
  
  \hypertarget{threat40}{}
  \item \textbf{T40: Pin für Zwei-Faktor-Authentisierung ist vorhersehbar}
  \begin{itemize}
    \item Zweck: Umgehen des intendierten Schutzmechanismus (Zwei-Faktor-Authentisierung)
    \item Möglichkeiten: Kryptografisch unsicheres Verfahren für die Zwei-Faktor-Authentisierung verwenden.
    \item Risiko: Mittel
    \item Mögliche Gegenmaßnahmen:
      \begin{itemize}
      	\item Time-based One-time Password algorithm (RFC 6238) verwenden.
      \end{itemize}
  \end{itemize}
    
  \hypertarget{threat41}{}
  \item \textbf{T41: Benutzer wird dazu gebracht, unintendiert die Zwei-Faktor-Authentisierung zu aktivieren}
  \begin{itemize}
    \item Zweck: Benutzer aus der Anwendung aussperren.
    \item Möglichkeiten: Benutzer wird mittels Clickjacking o.Ä. dazu gebracht, die Zwei-Faktor-Authentisierung in Verbindung mit einem zufälligen Secret zu aktivieren. Der Nutzer kennt dieses Secret nicht bzw. hat keine Gelegenheit, dieses zu speichern, da er die Aktivierung unbewusst durchführt.
    \item Risiko: Mittel
    \item Mögliche Gegenmaßnahmen:
      \begin{itemize}
      	\item Aktivierung der Zwei-Faktor-Authentisierung nur in Verbindung mit einem vom Secret abgeleiteten Pin ermöglichen.
      \end{itemize}
  \end{itemize}
  
\end{itemize}

\subsubsection{Mögliche Angriffe auf Single Sign-On}
\begin{itemize}

  \hypertarget{threat42}{}
  \item \textbf{T42: Entwenden des Klartextpassworts durch Administrator}
  \begin{itemize}
    \item Zweck: Umgehen der Verschlüsselung zum Abgreifen des Klartextpassworts.
    \item Möglichkeiten: Administrator legt den Netzdienst korrekt an. Nutzer hinterlegen ihre Zugangsdaten und verwenden Single Sign-On. Administrator ändert für den Netzdienst hinterlegte URL. Er kontrolliert den Host der neuen Ziel-URL und kann so die Zugangsdaten im Klartext abgreifen.
    \item Risiko: Mittel-Hoch
    \item Mögliche Gegenmaßnahmen:
      \begin{itemize}
      \item Keine Editierung von hinterlegten URLs zulassen
      \item Nutzer über visuelle Darstellung auf URL-Änderung hinweisen und explizit bestätigen lassen
      \end{itemize}
  \end{itemize}
  
  \hypertarget{threat43}{}
  \item \textbf{T43: Dateiinhalte über Zugangsdaten speichern}
  \begin{itemize}
    \item Zweck: Umgehung des Quota-Limits.
    \item Möglichkeiten: Angreifer speichert Dateien nicht über die vorgesehene Funktionalität, sondern codiert die Dateiinhalte in den Zugangsdaten.
    \item Risiko: Mittel
    \item Mögliche Gegenmaßnahmen:
      \begin{itemize}
      	\item Längenlimit auf die Zugangsdaten anwenden.
      \end{itemize}
  \end{itemize}
  
  \hypertarget{threat44}{}
  \item \textbf{T44: Auslesen von Klartext-Zugangsdaten im Falle einer Datenbank-Kompromittierung}
  \begin{itemize}
    \item Zweck: Erlangen von Klartext-Zugangsdaten zum Anmeldung auf Netzdiensten.
    \item Möglichkeiten: Angreifer erlangt Zugang auf Datenbank.
    \item Risiko: Mittel
    \item Mögliche Gegenmaßnahmen:
      \begin{itemize}
      	\item Zugangsdaten mit einem Secret verschlüsseln, das ausschließlich der Nutzer besitzt.
      \end{itemize}
  \end{itemize}
    
  \hypertarget{threat45}{}
  \item \textbf{T45: Entwenden des Klartextpassworts aus Speicher}
  \begin{itemize}
    \item Zweck: Umgehen der Verschlüsselung zum Abgreifen des Klartextpassworts.
    \item Möglichkeiten: Ein Bug wie Heartbleed, der Zugriff auf den Speicher eines Servers ermöglicht ausnutzen, um im Speicher vorhandene Passwörter auszulesen.
    \item Risiko: Gering
    \item Mögliche Gegenmaßnahmen:
      \begin{itemize}
      \item Passwörter nicht \enquote{unnötig} entschlüsseln. Nur dann entschlüsseln, wenn wirklich ein Single Sign-On erfolgen soll und dann ausschließlich die für diesen Sign-On erforderlichen Zugangsdaten entschlüsseln (so wenig Passwörter wie möglich in den Speicher laden).
      \end{itemize}
  \end{itemize}

\end{itemize}



\subsubsection{Mögliche Angriffe auf Zusatzfunktionalität: Nutzer kann eigenes Passwort ändern}

Diese Zusatzfunktionalität ist der Funktionalität des Logins zwar gleich, aber dadurch, dass diese Seite nur durch einen authentisierten Nutzer erreichbar ist, gelten hierfür nicht die selben Bedrohungen.
Wir sind uns bewusst, dass ein Angreifer, sofern er die Session hat, hier durch einen Brute-Force-Angriff die Möglichkeit hat, das Klartext-Passwort des Nutzers herauszufinden.

\subsubsection{Mögliche Angriffe beim Versand von E-Mails}
Der Angriffsvektor SMTP sollte ausweislich der Aufgabenstellung nicht betrachtet werden. Wir listen aus diesem Grund keine diesbezüglichen Bedrohungen auf.

\newcommand{\linktothreat}[2]{\texorpdfstring{\protect\hyperlink{#1}{#2}}{}}%

\section{Ignorierte Bedrohungen}
\subsection{Captchas durch menschliche Akteure lösen lassen}
Die Kosten für diese Dienstleistung machen sie unwirtschaftlich in Anbetracht unseres Applikationskontexts.

\subsection{HTTP bezogene Bedrohungen}
Die Bedrohungen \linktothreat{threat1}{T1}, \linktothreat{threat2}{T2}, \linktothreat{threat3}{T3}, \linktothreat{threat4}{T4}, \linktothreat{threat5}{T5} werden ignoriert.
Ausweislich der Aufgabenstellung sollen wir eine sichere Verbindung annehmen.

\subsection{Entwenden des Klartextpassworts durch Administrator}
Im Gespräch mit Herrn Prof. Dr. Peine wurde uns mitgeteilt, dass diese Bedrohung ignorierbar ist.

\section{Vorzunehmende Gegenmaßnahmen}

In der Tabelle \ref{threat-analysis:mitigation-table} können die Gegenmaßnahmen entnommen werden, die wir planen umzusetzen.

  \captionof{table}{Auflistung aller vorzunehmenden Gegenmaßnahmen mit Referenz auf Angriff, der verhindert/erschwert wird. Status beschreibt bereits durchgeführte Implementation.}
  \label{threat-analysis:mitigation-table}
  \begin{tabularx}{\linewidth}{
    |>{\hsize=0.7\hsize} X |
    >{\hsize=0.2\hsize} X |
    >{\hsize=0.1\hsize} X |
  }
  \hline
  \textbf{Gegenmaßnahme} & \textbf{Angriff} & \textbf{Status}\\ \hline
    Eingabevalidierung & \linktothreat{threat6}{T6}, \linktothreat{threat7}{T7}, \linktothreat{threat15}{T15}, \linktothreat{threat24}{T24}, \linktothreat{threat25}{T25}, \linktothreat{threat26}{T26} & \greencheckmark \\ \hline
    Ausgabekodierung & \linktothreat{threat6}{T6}, \linktothreat{threat25}{T25}, \linktothreat{threat31}{T31} & \greencheckmark \\ \hline
    PreparedStatements & \linktothreat{threat7}{T7}  & \greencheckmark \\ \hline
    \texttt{ALLOW\_LITERALS=NONE} aktivieren & \linktothreat{threat7}{T7}  & \greencheckmark \\ \hline
    Salt auf Passwörter anwenden & \linktothreat{threat8}{T8}  & \greencheckmark \\ \hline
    Hinzufügen von CSRF-Token zu Formularen/Requests & \linktothreat{threat9}{T9} & \greencheckmark \\ \hline
    Cookies mit privaten Schlüssel signieren & \linktothreat{threat10}{T10} & \greencheckmark \\ \hline
    Session-ID im UUID-Format persistieren & \linktothreat{threat10}{T10} & \greencheckmark \\ \hline
    Session an IP-Adresse binden & \linktothreat{threat10}{T10} & \greencheckmark \\ \hline
    Session-Lebenszeit stark begrenzen & \linktothreat{threat10}{T10} & \greencheckmark \\ \hline
    Aktionen inkl. User, IP-Adresse und Zeit loggen & \linktothreat{threat11}{T11} & \greencheckmark \\ \hline
    Zwei-Faktor-Authentisierung & \linktothreat{threat13}{T13}, \linktothreat{threat17}{T17} & \greencheckmark \\ \hline
    bcrypt als Hashing-Algorithmus verwenden & \linktothreat{threat15}{T15} & \greencheckmark \\ \hline
    Operationen unter konstanter, gleichbleibender Laufzeit - auch bei unterschiedlichen Ausführungssträngen - ausführen & \linktothreat{threat16}{T16}, \linktothreat{threat19}{T19}  & \greencheckmark \\ \hline
    Verbieten schwacher Passwörter (Mindestlänge sowie bekannte PWs nicht erlaubt) & \linktothreat{threat17}{T17} & \greencheckmark \\ \hline
    Captcha zwischenschalten, um Massenrequests entgegenzuwirken & \linktothreat{threat17}{T17}, \linktothreat{threat18}{T18}, \linktothreat{threat20}{T20} & \greencheckmark \\ \hline
    IP-Adresse bei auffälligem Verhalten sperren & \linktothreat{threat17}{T17} & \greencheckmark \\ \hline
    \enquote{Content Security Policy}-Header nutzen & \linktothreat{threat6}{T6}, \linktothreat{threat31}{T31} & \greencheckmark \\ \hline
    \enquote{X-XSS-Protection}-Header nutzen & \linktothreat{threat6}{T6}&  \greencheckmark \\ \hline
    Speicherplatzlimit auf Dateinamen und Kommentarfeld ausweiten & \linktothreat{threat21}{T21}  & \greencheckmark \\ \hline
    Hervorhebung der Nutzer-eigenen Datei & \linktothreat{threat22}{T22} & \greencheckmark \\ \hline
    Beim Download einer Datei HTTP-Header \texttt{Content-Type} auf \texttt{application/octet-stream} setzen & \linktothreat{threat23}{T23} & \greencheckmark \\ \hline
    Maximal-/Minimalwert für Session-Timeout definieren & \linktothreat{threat32}{T32}, \linktothreat{threat33}{T33}, \linktothreat{threat34}{T34} & \greencheckmark \\ \hline
    Return-URL via Regex überprüfen & \linktothreat{threat25}{T25} & \greencheckmark \\ \hline
    Funktionalität, die Return-URL benötigt, nur per POST akzeptieren & \linktothreat{threat25}{T25} & \greencheckmark \\ \hline
    Eigenen Logger implementieren, der potentiell gefährliche Zeichen filtert & \linktothreat{threat26}{T26} & \greencheckmark \\ \hline
    Passwort-Reset erst erlauben, wenn Nutzer initiales Passwort geändert hat & \linktothreat{threat38}{T38} & \greencheckmark \\ \hline
    Token mit ausreichender Länge zufällig generieren & \linktothreat{threat36}{T36} & \greencheckmark \\ \hline
    Token an IP-Addresse koppeln & \linktothreat{threat36}{T36} & \greencheckmark \\ \hline
    Zweiten Faktor auch beim Passwort-Zurücksetzen valdieren  & \linktothreat{threat37}{T37} & \greencheckmark \\ \hline
    Ausreichend langes Secret erstellen & \linktothreat{threat39}{T39} & \greencheckmark \\ \hline
    Zum Erstellen des Secrets SecureRandom verwenden & \linktothreat{threat39}{T39} & \greencheckmark \\ \hline
    Time-based One-time Password algorithm & \linktothreat{threat40}{T40} & \greencheckmark \\ \hline
    Aktivierung der Zwei-Faktor-Authentisierung nur in Verbindung mit einem vom Secret abgeleiteten Pin ermöglichen & \linktothreat{threat41}{T41} & \greencheckmark \\ \hline
    Längenlimit auf die Zugangsdaten anwenden & \linktothreat{threat43}{T43} & \greencheckmark \\ \hline
    Zugangsdaten mit einem Secret verschlüsseln, das ausschließlich der Nutzer besitzt  & \linktothreat{threat44}{T44} & \greencheckmark \\ \hline
    Zugangsdaten für externe Netzdienste nicht \enquote{unnötig} entschlüsseln & \linktothreat{threat45}{T45} & \greencheckmark \\ \hline
  \end{tabularx}
  
\subsection{HTTP Header}

In der folgenden Liste werden verschiedene sicherheitsrelevante \enquote{HTTP Headers} beschrieben, die uns helfen, einige der Gegenmaßnahmen zu realisieren.
Diese Werte werden empfohlen, um die \enquote{maximale} Sicherheit zu garantieren.
Entnommen haben wir diese Empfehlungen verschiedenen Seiten, darunter der \enquote{Mozilla's Enterprise Information Security}-Blog \autocite{Mozilla.SecureHeaders} und das \enquote{OWASP Secure Headers Project} \autocite{OWASP.SecureHeaders}, die wir beide als vertrauenswürdig betrachten.

\subsection{Referrer-Policy}
\begin{sloppypar}
\texttt{Referrer-Policy: origin-when-cross-origin, strict-origin-when-cross-origin}
\end{sloppypar}
Mit diesem Header kann gesteuert werden, welche Informationen als \enquote{Referrer} an die nächste besuchte Seite weitergegeben werden. \texttt{strict-origin-when-cross-origin} sorgt dafür, dass innerhalb der Seite die vollständige URL weitergegeben wird. Beim Besuchen einer anderen Seite wird nur die Domain weitergegeben. Wird dabei jedoch von HTTPS auf HTTP gewechselt, ist das Weitergeben jeder Information untersagt.

\subsection{X-Frame-Options}
\begin{sloppypar}
\texttt{X-Frame-Options: DENY}
\end{sloppypar}
Um zu verhindern, dass unsere Seite in einem \texttt{<frame>}, \texttt{<iframe>} oder \texttt{<object>} Element auf einer anderen Seite eingebettet wird, kann der Header \texttt{X-Frame-Options} verwendet werden. So kann Clickjacking verhindert werden. %So können \enquote{Clickjacking} sowie \texttt{<iframe>}-basierte CSRF-Angriffe verhindert werden.

\subsection{X-XSS-Protection}
\label{x:xss:protection}
\begin{sloppypar}
\texttt{X-XSS-Protection: 1; mode=block}
\end{sloppypar}
Durch den Header \texttt{X-XSS-Protection} wird der Browser (Chrome, Safari, Internet Explorer und Opera) angewiesen, eine \enquote{reflected XSS}-Erkennung zu versuchen.
Dieses Feature ist für älteren Browser interessant, die noch kein CSP implementieren.
Durch den Modus \enquote{block} wird das Laden der Seite abgebrochen.
Dieser Header dient daher als zusätzliches Auffangnetz, im Fachjargon oftmals \enquote{Defense in Depth} oder \enquote{castle approach} genannt, aber nicht als ausreichender Schutz.
Leider ist dies relativ einfach auszuhebeln oder sogar zu missbrauchen, daher fordern wir in unseren Annahmen explizit ein Ausschluss älterer Browser.
Die oben genannten Browser verwenden hierfür ein Blacklisting verschiedener Tags sowie gefährlicher Zeichen, bei welchen der \enquote{XSS-Auditor} anspringt und die potentielle XSS-Payload herausfiltert.
\footnote{Siehe \url{https://blog.innerht.ml/the-misunderstood-x-xss-protection/} für generelle Erklärung und Beispiel des Angriffs zum Entfernen der JQuery-Bibliothek}
Solch ein Blacklisting ist allerdings aufgrund verschiedener Möglichkeiten zum Deformieren des Payloads einfach auszuhebeln.
Darüber hinaus kann der XSS-Auditor auch missbraucht werden, um bestimmte JavaScript-Bibliotheken von einer Seite zu entfernen.
Chrome's/Chromium's XSS-Auditor\footnote{\url{https://bugs.chromium.org/p/chromium/issues/detail?id=825675}} entfernt beispielsweise \texttt{<script>}-Tags in dem HTML, wenn sie zeitgleich im Request-Query-String gefunden werden.

\subsection{X-Content-Type-Options}
\begin{sloppypar}
\texttt{X-Content-Type-Options: nosniff}
\end{sloppypar}
Mit diesem Header wird dem Browser verboten, \enquote{MIME sniffing}\footnote{Beim \enquote{MIME sniffing} versucht der Browser den richtigen MIME Type durch Betrachtung des Inhalts zu erraten, wenn er annimmt, dass der im \texttt{Content-Type} übermittelte Wert nicht korrekt ist.} zu betreiben. Er wird gezwungen den MIME Typ zu verwenden, der im \texttt{Content-Type} Header angegeben wurde.

\subsection{Content-Security-Policy}
\begin{sloppypar}
\texttt{Content-Security-Policy: default-src 'self'; script-src https://www.google.com/recaptcha/ https://www.gstatic.com/recaptcha/ http://localhost:9000/assets/js/ http://127.0.0.1:9000/assets/js/; frame-src https://www.google.com/recaptcha/; img-src 'self' data:}
\end{sloppypar}
Innerhalb diesem Header ist es möglich Richtlinien zu definieren, die den Zugriff auf verschiedene Resourcen kontrollieren.
Dadurch kann verhindert werden, dass JavaScript nicht von jeder URL geladen und ausgeführt werden darf, von welchen Quellen Bilder geladen werden dürfen, oder welches CSS eingebunden werden darf.
Dieser Header ist dem \enquote{X-XSS-Protection}-Header vorzuziehen.

%\subsection{X-Permitted-Cross-Domain-Policies}
%\begin{sloppypar}
%\texttt{X-Permitted-Cross-Domain-Policies: master-only}
%\end{sloppypar}

\printbibliography

% Can be used to add a list of acronyms with their description
%\glsaddall
%\deftranslation{to=German}{Acronyms}{Abkürzungsverzeichnis}
%\deftranslation{to=German}{Glossary}{Glossar}
\printacronyms[title=Abkürzungsverzeichnis,toctitle=Abkürzungsverzeichnis]
\printglossary[title=Glossar,toctitle=Glossar,type=main]

%\addcontentsline{toc}{chapter}{\listfigurename}
% Insert list of figures, if a figure has been added to document
\iftotalfigures
  \listoffigures
\fi

%s\addcontentsline{toc}{chapter}{\listtablename}
% \listoftables       % Tabellenverzeichnis

\end{document}
