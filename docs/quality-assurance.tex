\documentclass[12pt,DIV14,BCOR10mm,a4paper,twoside,parskip=half-,headsepline,headinclude,english,ngerman,bibliography=totocnumbered]{scrreprt}

\usepackage{hshhelper_base}

%%%%%%%%%%%%%%%%%%%%%%%%%%%%%%%%%%%%%%%%%%%%%%%%%%%%%%%%%%%%%%%%%%%%%%%%%%
\begin{document}    % hier gehts los
  \thispagestyle{empty} % Titelseite
\includegraphics[width=0.2\textwidth]{Wortmarke_WI_schwarz}

   {  ~ \sffamily
  \vfill
  {\Huge\bfseries Bericht: Maßnahmen zur Qualitätssicherung}
  \bigskip

  {\Large
  Dennis Grabowski, Julius Zint, Philip Matesanz, Torben Voltmer \\[2ex]
  Masterprojekt \enquote{Entwicklung und Analyse einer sicheren Web-Anwendung} \\
  Wintersemester 18/19
 \\[5ex]
   \today }
}
 \vfill

  ~ \hfill
  \includegraphics[height=0.3\paperheight]{H_WI_Pantone1665}

\vspace*{-3cm}

\tableofcontents  % Inhaltsverzeichnis

\chapter{Benutzte Methoden}
\section{Vier-Augen-Prinzip}
\section{Tests}

\subsection{Funktionale Tests}

Classcoverage, Branchcoverage, Linecoverage, Methodcoverage
End-to-End? Unittests? Wird DB getestet?
Welche Funktionen wurden nicht getestet?

\subsection{Tests zur Pruefung von sicherheitsrelevanten Funktionen}

Sessionmanagement?
Ist unser Cookie sicher und wirklich zufaellig?
Brute-force Schutz (Stichwort: "Firewall")
Authorization? Koennen Endpoints nur von den richtigen Privilegien angesurft werden?
Wird CSRF richtig genutzt?
Salted bcrypt richtig? Nutzen wir ausreichend Iterationen oder koennen wir mit einem Test das irgendwie knacken?
Kommt das Captcha zur richtigen Zeit? Ist das Captcha austricksbar?
Haben wir Quality rules fuer passwoerter?
Koennen wir kaputte Daten in die Forms eintragen?
Koennen wir irgendwelche Requests forgen?
Koennen SQL-Injections ausgefuehrt werden?
Setzen wir alle richtigen Security-Header?

\subsection{Penetrationstests}

Welche Penetrationstests wurden durchgefuehrt?
Gibt es ggf. automatisierbare Skripts?
Wenn ja, fuer welche Angriffe?

\section{Reviews}

Wer hat wann welchen Code reviewed?
Zeitliche Schaetzung?

\section{Statische Codeanalyse}

Welche Tools haben wir benutzt?
Wie haben wir sie eingeschaetzt?
Warum haben wir sie verwendet?
Wie haben wir sie verwendet?

Da innerhalb der Gruppe keine Praeferenzen bezüglich eines Werkzeugs fuer die statische Codeanalyse bestand, haben wir verschiedene ausprobiert.
Die Erfahrung aus anderen Sprachökosystemen (bspw. C++) zeigt, dass es dennoch ratsam ist, unterschiedliche Werkzeuge zu verwenden, um eine breitere Menge abzudecken und ggf. \enquote{False Positives} auszuschliessen.

Die Werkzeuge, die wir nach einer kurzen Evaluation ausgesucht haben, sind:

\begin{itemize}
  \item SpotBugs \autocite{SpotBugs} inkl. \enquote{find-sec-bugs}-Plugin \autocite{SpotBugs.FindSecBugs},
  \item Synopsis SecureAssist (ehemalig Cigital SecureAssist) \autocite{SecureAssist},
  \item SonarLint (ehemalig Sonarqube) \autocite{SonarLint}
\end{itemize}

\subsection{SpotBugs}

Wir haben die Version 3.1.8 verwendet, die am 2018-10-16 veroeffentlicht wurde.
Fuer \enquote{find-sec-bugs} wurde Version 1.8 verwendet.

Verwendetete Einstellungen:

\begin{itemize}
  \item Analyseaufwand: Maximal
  \item Minimalster Fehlerrang: 20 - Of Concern
  \item Alle Fehlerkategorien, Filter wurden aktiviert
\end{itemize}

\subsection{Synopsis SecureAssist}

Wir haben die Version 3.3.0 verwendet, welche im am 2018-01-29 veroeffentlicht wurde.
Da dieses Werkzeug im Betracht auf Sicherheitsluecken von beruehmten Sicherheitsexperten wie Gary McGraw entwickelt wurde, hoffen wir, dass dieses Werkzeug Luecken findet, die den anderen Werkzeugen nicht auffallen.

\subsubsection{Hardkodierte Passwoerter}

In unserer Testklasse \texttt{UserManagerTest} verwenden wir ein hardkodiertes Passwort, um zu ueberpruefen, ob ein Nutzer sein Passwort aendern kann.
Das sehen wir nicht als sicherheitsrelevanten Fehler an und ignorieren diesen daher.

\subsubsection{Query Injections}

Die Regel, die SecureAssist verwendet, um moegliche SQL Injections zu finden, scheint nur zu pruefen, ob ein SQL Statement als String verwendet wird, und ob es als \texttt{PreparedStatement} verwendet wird.
Leider ueberprueft es nicht, ob das als String angegebene SQL Statement ueberhaupt veraenderbar ist; beispielsweise durch Konkatenieren eines Parameters.
Das Werkzeug hat daher ein \enquote{False Positive} in unserer \texttt{DatabaseInitialization}-Klasse gefunden.
Die Statements in dieser Klasse koennen nicht von einem Nutzer durch Parameter angereichert werden.

\subsubsection{Information Leakage}

Das Ausgeben des Stacktrace im Falle einer Exception sieht das Werkzeug bereits Verletzung der Vertraulichkeit, welchem wir zustimmen.
Gluecklicherweise schuetzt Play einem in diesem Fall und gibt keine Stacktraces an einen Benutzer weiter, sofern die Applikation im \enquote{Production Mode} ausgefuehrt wird.
In diesem Fall wird eine Exception mit einer ID in das Applikationslog geschrieben.

\subsection{SonarLint}

An dem SonarLint-Plugin fuer die IntelliJ IDE konnten keine relevanten Einstellungen veraendert werden, daher wurden die Werkseinstellungen benutzt.

\section{Quelltextinspektion}

Haben wir ggf. irgendwelche Checklisten nutzen koennen?
Siehe

\chapter{Gefundene Maengel}

\printbibliography

% Can be used to add a list of acronyms with their description
%\glsaddall
%\deftranslation{to=German}{Acronyms}{Abkürzungsverzeichnis}
%\deftranslation{to=German}{Glossary}{Glossar}
\printacronyms[title=Abkürzungsverzeichnis,toctitle=Abkürzungsverzeichnis]
\printglossary[type=main]

%\addcontentsline{toc}{chapter}{\listfigurename}
\listoffigures      % Abbildungsverzeichnis

%s\addcontentsline{toc}{chapter}{\listtablename}
% \listoftables       % Tabellenverzeichnis

\end{document}
